\documentclass[10pt, a4paper]{report}

\usepackage[ngerman]{babel}
\usepackage[T1]{fontenc}

\usepackage[left=1cm, right=1cm, top=1cm, bottom=1cm]{geometry}
\pagestyle{empty}
\setlength\parindent{0pt}

\usepackage{graphicx}
\usepackage[table]{xcolor}
\usepackage{multirow}

\usepackage{hyperref}

\begin{document}

\tableofcontents

% Hier wird für jedes Protokoll die Kapitelüberschrift festgelegt
% PROTOKOLLINDEX wird mit dem Index des Protokolls / der Sitzungsnummer gefüllt
% WAHLPERIODE wird mit der entsprechenden Wahlperiode (19 oder 20) gefüllt
\chapter{1. Sitzung der 19. Wahlperiode}
\thispagestyle{empty}

% Hier werden die Reden gruppiert nach den Tagesordnungspunkten ausgegeben.
\section{Tagesordnungspunkt1}
\subsection{Rede ID19100100}

\textbf{Die Rede:}

% Bestehend aus Name, Partei/Fraktion und Bild
Name: Hermann Otto Solms, Partei: FDP, Fraktion: FDP

\begin{figure}[!ht]
\includegraphics[height=50px]{images/Hermann OttoSolms.jpg}
\end{figure}


Guten Morgen, liebe Kolleginnen und Kollegen! Nehmen Sie bitte Platz. Meine sehr verehrten Damen und Herren! Liebe Kolleginnen und Kollegen! Ich begrüße Sie zur konstituierenden Sitzung des 19. Deutschen Bundestages. Es entspricht der ständigen Übung, zu Beginn der konstituierenden Sitzung nach den Regelungen der bisherigen Geschäftsordnung des Deutschen Bundestages zu verfahren. § 1 Absatz 2 der Geschäftsordnung des Deutschen Bundestages sieht vor, dass das am längsten dem Bundestag angehörende Mitglied, das hierzu bereit ist, den Vorsitz übernimmt, bis der Deutsche Bundestag einen Präsidenten gewählt hat. Die Fraktion der AfD widerspricht diesem Verfahren und hat auf Drucksache 19/2 beantragt, einen Versammlungsleiter zu wählen, der die konstituierende Sitzung eröffnen soll. Über diesen Antrag lasse ich sofort abstimmen. Wer dem Antrag der Fraktion der AfD zustimmt, den bitte ich um sein Handzeichen. – Gegenstimmen? – Enthaltungen? – Der Antrag ist damit mit den Stimmen aller Fraktionen bei Zustimmung der AfD-Fraktion abgelehnt. Wir verfahren nunmehr entsprechend § 1 Absatz 2 der bisherigen Geschäftsordnung. Herr Dr. Wolfgang Schäuble ist das Mitglied, das dem Deutschen Bundestag am längsten angehört. Er hat jedoch auf das Amt des Alterspräsidenten verzichtet. Damit rückt das am zweitlängsten dem Bundestag angehörende Mitglied nach. Ich war von 1980 bis 2013, also 33 Jahre, Mitglied des Deutschen Bundestages. Ist jemand unter Ihnen, der dem Bundestag länger angehört? – Das scheint offenkundig nicht der Fall zu sein. Dann trifft es also tatsächlich zu, dass die Rolle des Alterspräsidenten auf mich zukommt. Es ist mir eine Ehre und Freude zugleich, den Vorsitz bis zur Amtsübernahme durch den neugewählten Präsidenten des Deutschen Bundestages zu übernehmen. Damit rufe ich nun den Tagesordnungspunkt 1 auf: Eröffnung der Sitzung durch den Alterspräsidenten Drucksache 19/2 

\textbf{Kommentare:}

Name: Hermann Otto Solms, Partei: FDP, Fraktion: FDP

\begin{figure}[!ht]
\includegraphics[height=50px]{images/Hermann OttoSolms.jpg}
\end{figure}


(Lachen bei Abgeordneten der AfD)
Name: Hermann Otto Solms, Partei: FDP, Fraktion: FDP

\begin{figure}[!ht]
\includegraphics[height=50px]{images/Hermann OttoSolms.jpg}
\end{figure}


(Heiterkeit)


\textbf{NLP-Informationen:}

\textit{DDC-Kategorie der Rede:}

Politikwissenschaft

\textit{Named Entities:}

Personen:

Dr. Wolfgang Schäuble, 

Orte:

Amt des Alterspräsidenten, 

Organisationen:

Deutschen Bundestages, Deutschen Bundestages, Bundestag, Deutsche Bundestag, AfD, AfD, AfD-Fraktion, Deutschen Bundestag, Bundestag, Bundestag, Deutschen Bundestages, 

\textit{Sentiment auf Satzebene:}

0.75, 0.0, 0.4186, 0.7177, 0.0, 0.0, 0.2023, -0.1027, 0.128, 0.4215, 0.0, 0.0, 0.1531, 0.0, 0.0, -0.1531, 0.0, 0.0, 0.0, 0.0, 0.0, 0.7506, 0.0, 0.0, 0.0, 
\section{Tagesordnungspunkt2}
\subsection{Rede ID19100400}

\textbf{Die Rede:}

% Bestehend aus Name, Partei/Fraktion und Bild
Name: Bernd Baumann, Partei: AfD, Fraktion: AfD

\begin{figure}[!ht]
\includegraphics[height=50px]{images/BerndBaumann.jpg}
\end{figure}


Herr Präsident! Meine Damen und Herren! Immer deutlicher zeigte sich im Verlauf dieses Jahres, dass die AfD in den Bundestag einziehen würde und dass sie auch den Alterspräsidenten stellen würde, weil sie neben vielen jungen Abgeordneten eben auch den ältesten Abgeordneten mit an Bord hatte. Als dies deutlich wurde, änderten Sie hier im alten Bundestag ganz schnell die Geschäftsordnung – knapp vor der Wahl, zwei Wochen vor Ende Sitzungsperiode. Meine Damen und Herren, das war ein so durchsichtiges Manöver; das lassen wir Ihnen hier nicht durchgehen! Eine List, mit der Sie die AfD ausgrenzen wollten! Das kann so nicht weitergehen! Die AfD hatte gerade den 13. Wahlsieg in Folge erzielt – eine Triumphserie bei Landtagswahlen, die so zuvor keine Partei nach ihrer Gründung erzielt hatte. Der alte Bundestag aber änderte die Geschäftsordnung plötzlich so, dass nicht mehr der älteste Abgeordnete, sondern der mit der längsten Dienstzeit die erste Sitzung als Alterspräsident eröffnen sollte. Und das würde dann nicht mehr ein AfD-Abgeordneter sein. Sie begründeten dies damit, dass nur der dienstälteste Abgeordnete eine korrekte Sitzungsleitung sicherstellen könne. Das war das Argument. Aber, meine Damen und Herren, seit 1848 ist in Deutschland Tradition, dass die konstituierende Sitzung natürlich vom ältesten Mitglied der Versammlung eröffnet wird. Das war eine Tradition von der Frankfurter Paulskirche bis Gustav Stresemann, von Adenauer über Brandt bis Kohl, ja bis zu der ersten Regierung Merkel. Alle Reichstage, alle Bundestage konnten dem Argument „Dienstalter“ nichts abgewinnen. Das ist ja auch klar. In 150 Jahren Parlamentsgeschichte blieb die Regel des Alterspräsidenten unangetastet. Unangetastet? Es gab eine Ausnahme: 1933 hat Hermann Göring die Regel gebrochen, weil er politische Gegner ausgrenzen wollte, damals Clara Zetkin. Wollen Sie sich auf solch schiefe Bahn begeben? Kommen Sie zurück auf die Linie der großen deutschen Demokraten! Dazu fordere ich Sie hier auf. Denn stets eröffnete der Parlamentarier, der aufgrund von Lebenserfahrung und Altersweisheit Versammlungen besonders umsichtig eröffnen konnte. Das waren die Idee und die Tradition über 150 Jahre. Denn zur bloßen Eröffnung ist ja kein geballtes Geschäftsordnungswissen vonnöten. Es wird sofort ein Versammlungspräsident gewählt, der versiert in der Sache ist. Auch deshalb ist Ihr Argument, der Dienstälteste müsse aus fachlichen Gründen eröffnen, schlichtweg unsinnig, meine Damen und Herren. Selbst die Presse, die uns ja nicht durchgängig gewogen ist, warnte, dass dieses Manöver gegen die AfD durchsichtig sei, und sprach von Lex AfD. Der „Tagesspiegel“ sprach von „Trickserei“. „Focus“ resümierte: Die Entscheidung wirft kein gutes Licht auf die parlamentarische Kultur in Deutschland. Recht hat er, meine Damen und Herren! Wie groß, frage ich Sie, muss die Angst vor der AfD und ihren Wählern sein, wenn Sie zu solchen Mitteln greifen? Deswegen fordere ich Sie auf: Kehren Sie um! Vertrauen Sie wie wir den bewährten Traditionen des deutschen Parlamentarismus! Stimmen Sie unserem Antrag zu! Meine Damen und Herren, nehmen Sie zur Kenntnis: Der alte Bundestag, in dem Sie alles untereinander regeln und die Konkurrenz wegdrücken konnten wie hier bei der Frage des Alterspräsidenten, wurde abgewählt. Das Volk hat entschieden. Nun beginnt eine neue Epoche. Von dieser Stunde an werden hier Themen neu verhandelt, nicht nur Ihre Manöver und Tricks bei der Geschäftsordnung, sondern künftig auch Euro, gigantische Schuldenübernahmen, riesige Einwanderungszahlen, offene Grenzen und immer brutalere Kriminalität auf unseren Straßen, meine Damen und Herren. Das Wort hat jetzt der Kollege Jan Korte von der Fraktion Die Linke. 

\textbf{Kommentare:}

Name: Bernd Baumann, Partei: AfD, Fraktion: AfD

\begin{figure}[!ht]
\includegraphics[height=50px]{images/BerndBaumann.jpg}
\end{figure}


(Christian Lindner [FDP]: Stimmt gar nicht!)
Name: Bernd Baumann, Partei: AfD, Fraktion: AfD

\begin{figure}[!ht]
\includegraphics[height=50px]{images/BerndBaumann.jpg}
\end{figure}


(Beifall bei der AfD)
Name: Bernd Baumann, Partei: AfD, Fraktion: AfD

\begin{figure}[!ht]
\includegraphics[height=50px]{images/BerndBaumann.jpg}
\end{figure}


(Beifall bei der AfD – Dr. Marco Buschmann [FDP]: Traditionen wollten Sie doch direkt brechen!)
Name: Bernd Baumann, Partei: AfD, Fraktion: AfD

\begin{figure}[!ht]
\includegraphics[height=50px]{images/BerndBaumann.jpg}
\end{figure}


(Zurufe von der SPD)
Name: Bernd Baumann, Partei: AfD, Fraktion: AfD

\begin{figure}[!ht]
\includegraphics[height=50px]{images/BerndBaumann.jpg}
\end{figure}


(Zuruf von der FDP: Mir kommen die Tränen!)
Name: Bernd Baumann, Partei: AfD, Fraktion: AfD

\begin{figure}[!ht]
\includegraphics[height=50px]{images/BerndBaumann.jpg}
\end{figure}


(Michael Theurer [FDP]: Rabulisten!)
Name: Bernd Baumann, Partei: AfD, Fraktion: AfD

\begin{figure}[!ht]
\includegraphics[height=50px]{images/BerndBaumann.jpg}
\end{figure}


(Beifall bei der AfD – Martin Schulz [SPD]: Da kennt ihr euch ja aus!)
Name: Bernd Baumann, Partei: AfD, Fraktion: AfD

\begin{figure}[!ht]
\includegraphics[height=50px]{images/BerndBaumann.jpg}
\end{figure}


(Beifall bei der AfD – Zuruf von der FDP: Zur Sache!)
Name: Bernd Baumann, Partei: AfD, Fraktion: AfD

\begin{figure}[!ht]
\includegraphics[height=50px]{images/BerndBaumann.jpg}
\end{figure}


(Beifall bei der LINKEN)


\textbf{NLP-Informationen:}

\textit{DDC-Kategorie der Rede:}

Literaturen germanischer Sprachen; Deutsche Literatur

\textit{Named Entities:}

Personen:

Gustav Stresemann, Adenauer, Brandt, Hermann Göring, Clara Zetkin, Lex AfD., Tricks, Jan Korte, 

Orte:

Landtagswahlen, Deutschland, Frankfurter Paulskirche, Deutschland, 

Organisationen:

Präsident!, Bundestag, Abgeordneten, Abgeordneten, Bundestag, Bundestag, deutschen Demokraten, AfD, Tagesspiegel, Bundestag, Die Linke, 

\textit{Sentiment auf Satzebene:}

0.0, 0.0, -0.2023, -0.25, 0.0, 0.0, -0.5093, 0.0, 0.0, 0.6124, -0.0536, 0.0, 0.6249, 0.0, 0.25, 0.4215, 0.0, 0.3182, 0.0, 0.0, -0.6124, 0.0, 0.3382, 0.0, 0.0, 0.4215, 0.0, 0.25, -0.2732, 0.0, 0.0, 0.0, 0.4767, 0.2481, -0.296, 0.0, 0.0, 0.8221, 0.0, 0.0, -0.3182, 0.0, 0.4404, 0.3818, 0.0, 
\subsection{Rede ID19100500}

\textbf{Die Rede:}

% Bestehend aus Name, Partei/Fraktion und Bild
Name: Jan Korte, Partei: DIE LINKE., Fraktion: DIE LINKE.

\begin{figure}[!ht]
\includegraphics[height=50px]{images/JanKorte.jpg}
\end{figure}


Herr Präsident! Liebe Kolleginnen und Kollegen! Sehr geehrte Damen und Herren! Bei aller Kritik an der parlamentarischen Demokratie, der Geschäftsordnung und vielem anderen mehr will ich zumindest an eines erinnern: Wir sitzen hier aufgrund von freien Wahlen. Das haben Millionen andere nicht. Und wir sind dem Grundgesetz verpflichtet, das nicht einfach irgendein Gesetz ist, sondern das die humane und demokratische Antwort auf die Verheerung und die Leichenberge des NS-Faschismus gewesen ist. Das gilt es jeden Tag zu verteidigen. Aber es gibt natürlich auch berechtigte Kritik an unserem Verfahren hier, am Parlamentarismus. Dieser müssen wir uns stellen, und wir müssen das bessern. Deswegen haben wir als Linke zwei Änderungsanträge eingebracht, die ich kurz vorstellen will. Wir wollen das Ganze lebendiger machen. Wir brauchen natürlich wirklich eine bessere Kontrolle des Regierungshandelns, und wir müssen die Abgeordnetenrechte und den Bundestag als solchen stärken. Jeder, der einmal die Regierungsbefragung und die Fragestunde gesehen hat – oder besser gesagt: ertragen musste –, entweder live hier oder bei Phoenix, weiß, dass wir diese grundlegend reformieren müssen. Dazu hat der geschätzte Präsident Lammert viel Richtiges gesagt. Dazu gehört zwingend, dass die Fraktionen, besonders natürlich die Oppositionsfraktionen, hier Themen vorschlagen können, die zum Beispiel – so sie denn gewählt ist – der Bundeskanzlerin und der Bundesregierung sehr unangenehm sind. Das ist ganz zentral. Ich will eines sagen: Was wirklich nicht geht, ist, dass es eine Befragung der Bundeskanzlerin lediglich in der Bundespressekonferenz gibt. Das geht nicht, liebe Kolleginnen und Kollegen. Ich glaube, wir sollten das auch noch aus einem anderen Grunde tun: Erstens wollen wir hier besser werden: besser arbeiten, bessern streiten, wofür man übrigens Sachargumente und viele Stunden Zeit braucht. Ich glaube ferner, dass es auch politisch sinnvoll ist, heute ein Zeichen an diejenigen zu senden, die sich abgewandt haben, die gar keine Teilhabe an dieser Demokratie mehr haben, die nicht mehr wählen gehen, die damit nichts mehr zu tun haben wollen. Dafür ist es, finde ich, das Mindeste, zu versuchen, es besser zu machen. In diesem Zusammenhang komme ich zum zweiten Antrag, den meine Fraktion, Die Linke, einbringt. Wir möchten – so wie es im Grundgesetz vorgeschrieben ist – sofort, also heute, die vier im Grundgesetz zwingend vorgeschriebenen Ausschüsse des Bundestages einsetzen. Das sind der Auswärtige Ausschuss, der Europaausschuss, der Verteidigungsausschuss und der Petitionsausschuss. Warum? Erstens. Wir alle hier – deswegen sitzen wir zusammen – sind gewählt. Das hat nichts mit Koalitionsverhandlungen zu tun. Der Bundestag ist gewählt! Wir sollten mit der Arbeit beginnen, liebe Kolleginnen und Kollegen. Zweitens. Wählerinnen und Wähler haben logischerweise gerade in diesen Zeiten einen Anspruch darauf, dass das Parlament arbeitet, dass wir miteinander streiten und ringen. Wir müssen in den nächsten Monaten Auslandseinsätze der Bundeswehr verlängern, besser: ablehnen. Aber dafür muss man beraten und Expertise einholen. Das ist doch eigentlich völlig logisch. Ob Jamaika nun kommt oder nicht, werden wir alles sehen. Aber die Verhandlungen können doch nicht allen Ernstes dazu führen, dass wir monatelang den Bundestag in Geiselhaft nehmen, bis Sie Ihre Befindlichkeiten in Ihrer Koalition gelöst haben; das geht nicht. Lassen Sie uns als Bundestag arbeiten und streiten! Der letzte Punkt, den ich ansprechen möchte, ist: Das hat auch etwas mit dem Selbstbild des Bundestages zu tun. Sind wir als Abgeordnete so selbstbewusst, zu sagen: „Die innere Organisation des Bundestages hängt nicht an Koalitionsverhandlungen, egal wie sie aussehen“? Ich finde, dieses Selbstbewusstsein sollten wir haben, liebe Kolleginnen und Kollegen. Deswegen zielen unsere Anträge auf die Stärkung des Bundestages, auf mehr Diskurs in diesem Bundestag und heute vor allem als ein Zeichen an diejenigen, die sich abgewandt haben, aber für die wir auch hier sitzen und die wir zurückgewinnen müssen, für demokratische Lösungen, für Streit, für die Menschenwürde und für das Kenntlichmachen, wo in diesem Haus die fundamentalen Unterschiede liegen. Dafür ist erforderlich, dass wir als Bundestag beginnen, zu arbeiten. Das müsste doch eigentlich auf der Hand liegen. Deshalb bitte ich um Zustimmung für unsere Änderungsanträge. Das Wort hat jetzt Herr Kollege Michael Grosse-Brömer von der CDU/CSU-Fraktion. 

\textbf{Kommentare:}

Name: Jan Korte, Partei: DIE LINKE., Fraktion: DIE LINKE.

\begin{figure}[!ht]
\includegraphics[height=50px]{images/JanKorte.jpg}
\end{figure}


(Beifall bei der LINKEN sowie bei Abgeordneten der SPD, der FDP und des BÜNDNISSES 90/DIE GRÜNEN)
Name: Jan Korte, Partei: DIE LINKE., Fraktion: DIE LINKE.

\begin{figure}[!ht]
\includegraphics[height=50px]{images/JanKorte.jpg}
\end{figure}


(Beifall bei der LINKEN sowie bei Abgeordneten der SPD, der AfD, der FDP und des BÜNDNISSES 90/DIE GRÜNEN)
Name: Jan Korte, Partei: DIE LINKE., Fraktion: DIE LINKE.

\begin{figure}[!ht]
\includegraphics[height=50px]{images/JanKorte.jpg}
\end{figure}


(Beifall bei der LINKEN, der SPD und dem BÜNDNIS 90/DIE GRÜNEN sowie bei Abgeordneten der CDU/CSU, der AfD und der FDP)
Name: Jan Korte, Partei: DIE LINKE., Fraktion: DIE LINKE.

\begin{figure}[!ht]
\includegraphics[height=50px]{images/JanKorte.jpg}
\end{figure}


(Beifall bei der LINKEN, der SPD, der AfD und dem BÜNDNIS 90/DIE GRÜNEN sowie bei Abgeordneten der FDP)
Name: Jan Korte, Partei: DIE LINKE., Fraktion: DIE LINKE.

\begin{figure}[!ht]
\includegraphics[height=50px]{images/JanKorte.jpg}
\end{figure}


(Beifall bei der CDU/CSU sowie bei Abgeordneten der FDP)
Name: Jan Korte, Partei: DIE LINKE., Fraktion: DIE LINKE.

\begin{figure}[!ht]
\includegraphics[height=50px]{images/JanKorte.jpg}
\end{figure}


(Beifall bei der LINKEN sowie bei Abgeordneten der SPD und der AfD)
Name: Jan Korte, Partei: DIE LINKE., Fraktion: DIE LINKE.

\begin{figure}[!ht]
\includegraphics[height=50px]{images/JanKorte.jpg}
\end{figure}


(Beifall bei der LINKEN sowie bei Abgeordneten der AfD)


\textbf{NLP-Informationen:}

\textit{DDC-Kategorie der Rede:}

Literaturen germanischer Sprachen; Deutsche Literatur

\textit{Named Entities:}

Personen:

Lammert, Michael Grosse-Brömer, 

Orte:

humane, bessern, Bundespressekonferenz, Erstens, Mindeste, Auswärtige Ausschuss, Warum, Erstens, Zweitens, Jamaika, Kenntlichmachen, Unterschiede, 

Organisationen:

Präsident!, NS-Faschismus, Bundestag, Phoenix, Die Linke, Bundestages, Bundestag, Bundeswehr, Bundestag, Bundestag, Bundestages, Bundestages, Bundestages, Bundestag, Bundestag, CDU, 

\textit{Sentiment auf Satzebene:}

0.0, 0.7177, 0.0, -0.128, 0.4767, 0.0, -0.016, 0.2023, -0.128, 0.4404, -0.1531, 0.4215, 0.6908, 0.5859, 0.4215, -0.3804, 0.0, 0.0, -0.357, 0.8658, 0.6808, 0.4767, 0.0, 0.4019, 0.0, 0.0, 0.0, 0.0, 0.0, 0.0, 0.4588, 0.0, 0.2023, 0.3612, 0.4215, 0.4391, 0.0, 0.5719, 0.0, -0.4574, 0.4019, 0.0, 0.0, 0.4588, 0.8074, 0.0, 0.4939, 0.4404, 0.0, 
\subsection{Rede ID19100600}

\textbf{Die Rede:}

% Bestehend aus Name, Partei/Fraktion und Bild
Name: Michael Grosse-Brömer, Partei: CDU, Fraktion: CDU

\begin{figure}[!ht]
\includegraphics[height=50px]{images/MichaelGrosse-Brömer.jpg}
\end{figure}


Herr Präsident! Sehr verehrte Gäste! Meine lieben Kolleginnen und Kollegen! Lieber Kollege Schneider, ich verstehe, dass man mit 20 Prozent als Bundestagswahlergebnis nach einem Schuldigen sucht. Ich empfehle, nicht immer nur ins Kanzleramt zu gucken. Suchen Sie im Willy-Brandt-Haus! Dort ist der Erfolg größer, dass Sie einen Schuldigen finden. Das geht dann vor allen Dingen auch schneller. Davon abgesehen stehen wir heute am Beginn einer Legislaturperiode. Wir haben bald – sofort nach der Konstituierung – Arbeitsfähigkeit hergestellt. Wir haben 289 Kolleginnen und Kollegen als neu gewählte Abgeordnete unter uns, die in der letzten Legislaturperiode diesem Bundestag nicht angehört haben. Zwei Fraktionen sind neu ins Parlament gezogen. Diese neuen Abgeordneten werden jetzt mit teils umfangreichen Änderungsanträgen zur Geschäftsordnung begrüßt. Ich halte das weder für besonders stilvoll noch für sachlich gerechtfertigt und fair. Die neuen Kolleginnen und Kollegen hatten weder die Gelegenheit, sich mit Ihren Anträgen auseinanderzusetzen, noch hatten sie Gelegenheit, jemals live eine Regierungsbefragung oder das, was Sie ändern wollen, überhaupt zur Kenntnis zu nehmen oder zu erfahren. Deswegen sagen wir: Ungeachtet der Möglichkeit, etwas zu ändern, wollen wir nicht, dass heute über die Anträge abgestimmt wird. Wir wollen sie überweisen, damit sie in Ruhe und mit der gebotenen Sachlichkeit beraten werden können. Insofern soll sich zunächst der Ältestenrat damit beschäftigen. Weil wir am Anfang der Beratungen stehen, möchte ich nur ein paar Anmerkungen machen. Erstens: zum Änderungsantrag der AfD bezüglich des Alterspräsidenten. Hermann Otto Solms hat heute bewiesen, dass es sinnvoll ist, nicht nur auf Lebensalter zu setzen, sondern auch auf politische Erfahrung. Deswegen sehe ich da keinen Änderungsbedarf. Zweitens: zu den Anträgen von SPD und Linken zu Änderungen bei der Regierungsbefragung. Sie wollen eine lebendige und informative Regierungsbefragung. Gar keine Frage, das wollen wir auch. Das Fragerecht muss aber Instrument der parlamentarischen Kontrolle bleiben und sollte nicht Kampfinstrument der Opposition werden. Ich glaube, es liegt in unserem gemeinsamen Interesse, dass wir sachlich argumentieren und nicht versuchen, nur persönlich zu attackieren. Deswegen habe ich da Bedenken. Aber um auch das klar zu sagen: Wir sind bereit, darüber zu reden – in der gebotenen Sachlichkeit und nicht im Hauruckverfahren. Ich kann verstehen, Herr Kollege Schneider, dass die SPD möglichst schnell in die Opposition will. Wir werden Sie dabei unterstützen, aber dieses Antrags bedarf es dazu nicht. Drittens: zum Antrag der Linken auf unmittelbare Einsetzung der vier im Grundgesetz genannten Ausschüsse. Ich möchte betonen: Dieser Bundestag, der 19. Deutsche Bundestag, ist ab heute jederzeit in der Lage, zu arbeiten und Entscheidungen zu treffen, er ist arbeitsfähig und entscheidungsfähig, um das einmal klar zu sagen. Sie haben hier suggeriert, wir müssten diese Ausschüsse einsetzen. Das müssen wir nicht. Es gibt keine Fristen im Grundgesetz oder sonst wo, die uns vorgeben, diese Ausschüsse heute einzusetzen. Wir haben eine parlamentarische Gepflogenheit, Ausschüsse spiegelbildlich zu den vorhandenen Ministerien einzusetzen. Wenn die aber noch nicht da sind, wird es schwierig mit dieser parlamentarischen Gepflogenheit. Deswegen bin ich dafür: Wir machen es so, wie es sich bewährt hat. Ich denke, der parlamentarischen Praxis ist hier erneut Vorrang einzuräumen. Einzelne Fachausschüsse vorzuziehen, dieses Schrittes bedarf es nicht. Abschließend zum Antrag der AfD zu Minderheitenrechten. Aufgrund einer starken Großen Koalition bestand in der 18. Wahlperiode Anlass, darüber nachzudenken, ob Minderheitenrechte in ausreichendem Maße gewährleistet werden können, zugegeben bei einer damals recht kleinen Opposition. Die Minderheitenrechte waren Reaktion auf Mehrheitsverhältnisse, und es gibt deshalb auch keinen Grund, sie zu verlängern. Die ­GroKo ist nicht mehr vorhanden. Interessant ist ja auch, dass die AfD das Quorum auf 65 Abgeordnete senken will. Möglicherweise ist das die vorbeugende Reaktion auf weitere Austritte aus Ihrer Fraktion. Das weiß ich nicht. Mal sehen, ob bei 65 Schluss ist. Sie haben ja noch ausreichend Zeit. Meine Damen und Herren, liebe Kolleginnen und Kollegen, die Anträge, die wir hier vorgestellt bekommen haben, gehören nicht in die konstituierende Sitzung. Wir haben viel Wichtigeres zu tun. Sie gehören in Ruhe beraten, damit man sich auch fachlich damit auseinandersetzen kann. Dafür sorgen wir als CDU/CSU-Bundestagsfraktion. Vielen Dank. Das Wort hat jetzt der Kollege Dr. Marco Buschmann von der FDP-Fraktion. 

\textbf{Kommentare:}

Name: Michael Grosse-Brömer, Partei: CDU, Fraktion: CDU

\begin{figure}[!ht]
\includegraphics[height=50px]{images/MichaelGrosse-Brömer.jpg}
\end{figure}


(Widerspruch bei der SPD)
Name: Michael Grosse-Brömer, Partei: CDU, Fraktion: CDU

\begin{figure}[!ht]
\includegraphics[height=50px]{images/MichaelGrosse-Brömer.jpg}
\end{figure}


(Dr. Petra Sitte [DIE LINKE]: Was ist denn mit dem Petitionsrecht?)
Name: Michael Grosse-Brömer, Partei: CDU, Fraktion: CDU

\begin{figure}[!ht]
\includegraphics[height=50px]{images/MichaelGrosse-Brömer.jpg}
\end{figure}


(Beifall bei Abgeordneten der CDU/CSU und der FDP)
Name: Michael Grosse-Brömer, Partei: CDU, Fraktion: CDU

\begin{figure}[!ht]
\includegraphics[height=50px]{images/MichaelGrosse-Brömer.jpg}
\end{figure}


(Matthias W. Birkwald [DIE LINKE]: Wer kontrolliert denn die?)
Name: Michael Grosse-Brömer, Partei: CDU, Fraktion: CDU

\begin{figure}[!ht]
\includegraphics[height=50px]{images/MichaelGrosse-Brömer.jpg}
\end{figure}


(Martin Schulz [SPD]: Ach!)
Name: Michael Grosse-Brömer, Partei: CDU, Fraktion: CDU

\begin{figure}[!ht]
\includegraphics[height=50px]{images/MichaelGrosse-Brömer.jpg}
\end{figure}


(Martin Schulz [SPD]: Wir helfen gerne!)
Name: Michael Grosse-Brömer, Partei: CDU, Fraktion: CDU

\begin{figure}[!ht]
\includegraphics[height=50px]{images/MichaelGrosse-Brömer.jpg}
\end{figure}


(Beifall bei der CDU/CSU und der FDP)
Name: Michael Grosse-Brömer, Partei: CDU, Fraktion: CDU

\begin{figure}[!ht]
\includegraphics[height=50px]{images/MichaelGrosse-Brömer.jpg}
\end{figure}


(Carsten Schneider [Erfurt] [SPD]: Wir sind es schon!)
Name: Michael Grosse-Brömer, Partei: CDU, Fraktion: CDU

\begin{figure}[!ht]
\includegraphics[height=50px]{images/MichaelGrosse-Brömer.jpg}
\end{figure}


(Beifall bei der FDP sowie bei Abgeordneten der CDU/CSU)
Name: Michael Grosse-Brömer, Partei: CDU, Fraktion: CDU

\begin{figure}[!ht]
\includegraphics[height=50px]{images/MichaelGrosse-Brömer.jpg}
\end{figure}


(Ulli Nissen [SPD]: Wie viel haben Sie denn verloren?)
Name: Michael Grosse-Brömer, Partei: CDU, Fraktion: CDU

\begin{figure}[!ht]
\includegraphics[height=50px]{images/MichaelGrosse-Brömer.jpg}
\end{figure}


(Beifall bei der CDU/CSU sowie bei Abgeordneten der FDP)
Name: Michael Grosse-Brömer, Partei: CDU, Fraktion: CDU

\begin{figure}[!ht]
\includegraphics[height=50px]{images/MichaelGrosse-Brömer.jpg}
\end{figure}


(Jan Korte [DIE LINKE]: Müssen wir aber!)
Name: Michael Grosse-Brömer, Partei: CDU, Fraktion: CDU

\begin{figure}[!ht]
\includegraphics[height=50px]{images/MichaelGrosse-Brömer.jpg}
\end{figure}


(Widerspruch bei der LINKEN – Matthias W. Birkwald [DIE LINKE]: Es gibt kein Ministerium für die Opposition, Herr Kollege!)
Name: Michael Grosse-Brömer, Partei: CDU, Fraktion: CDU

\begin{figure}[!ht]
\includegraphics[height=50px]{images/MichaelGrosse-Brömer.jpg}
\end{figure}


(Zurufe von der SPD und der LINKEN: Oh!)
Name: Michael Grosse-Brömer, Partei: CDU, Fraktion: CDU

\begin{figure}[!ht]
\includegraphics[height=50px]{images/MichaelGrosse-Brömer.jpg}
\end{figure}


(Beifall bei der CDU/CSU, der SPD, der FDP und dem BÜNDNIS 90/DIE GRÜNEN)


\textbf{NLP-Informationen:}

\textit{DDC-Kategorie der Rede:}

Bräuche, Etikette, Folklore

\textit{Named Entities:}

Personen:

Schneider, Erstens, Hermann Otto Solms, Gar, Schneider, Minderheitenrechte, 

Orte:

Willy-Brandt-Haus, Zweitens, Drittens, Mehrheitsverhältnisse, Wichtigeres, 

Organisationen:

Bundestag, Abgeordneten, SPD, Linken, SPD, Bundestag, CDU, FDP-Fraktion, 

\textit{Sentiment auf Satzebene:}

0.0, 0.4186, 0.5093, -0.25, 0.3818, 0.0, 0.2023, 0.4019, 0.0, 0.0, 0.4404, 0.4404, 0.6003, 0.3707, 0.5423, 0.5046, 0.5106, 0.0, 0.4019, 0.0, 0.6124, 0.0, 0.0, 0.4215, 0.0, 0.0762, 0.5994, 0.128, 0.4767, 0.4215, 0.4019, 0.0, 0.6124, 0.0, 0.0, 0.0, 0.0, -0.34, 0.0, 0.3182, 0.0, 0.0, 0.0, 0.0, 0.8442, 0.0, 0.0, 0.25, 0.0, 0.0, 0.0, 0.128, 0.4588, 0.4019, 0.3818, -0.2263, 0.0, 0.0, 
\subsection{Rede ID19100700}

\textbf{Die Rede:}

% Bestehend aus Name, Partei/Fraktion und Bild
Name: Marco Buschmann, Partei: FDP, Fraktion: FDP

\begin{figure}[!ht]
\includegraphics[height=50px]{images/MarcoBuschmann.jpg}
\end{figure}


Herr Präsident! Meine lieben Kolleginnen und Kollegen! Die konstituierende Sitzung des Deutschen Bundestages ist eines der schönsten Hochämter der deutschen Demokratie. Es gibt kaum einen größeren, einen bedeutenderen Anlass als diesen, wo wir uns gemeinsam hinter dem Prinzip der repräsentativen Demokratie versammeln. So groß, wie dieser Anlass ist, so klein muss man denken, wenn man diese Sitzung als Bühne für die Sucht nach Aufmerksamkeit und Öffentlichkeit missbrauchen möchte. Genau das tun aber alle Antragsteller. Alle wollen sie ihre eingeübten Rollen spielen. Die AfD geriert sich – ganz bewährt – als Opfer einer finsteren Verschwörung. Das kennen wir zur Genüge, das haben wir oft genug erlebt. Ich wollte dazu eigentlich gar nichts mehr sagen. Aber, lieber Herr Kollege Dr. Baumann, indem Sie sich hier ernsthaft mit den O pfern Hermann Görings verglichen haben, haben Sie sich an Geschmacklosigkeit wieder selbst übertroffen. Auch Die Linke hat ihre bewährte Rolle. Sie gibt hier die Scheinheilige. Herr Korte trägt vor, es ginge darum, das Parlament schnell arbeitsfähig zu machen. Sie kennen die Praxis des Parlamentarismus in Deutschland ganz genau. Wir wissen heute nicht, welche Ausschüsse es insgesamt geben wird, wie groß sie sein werden, welche Kollegen wohin gehen müssen. Dass das in den Fraktionen natürlich besprochen werden muss, ist Ihnen bekannt. Es geht Ihnen um nichts anderes, als hier wieder ein Stück Aufmerksamkeit für sich in Anspruch zu nehmen. Darin unterscheiden Sie sich mit Ihrem Anliegen kein bisschen von der AfD. Wissen Sie, dass die extreme Rechte und die extreme Linke versuchen, diese Bühne zu missbrauchen, war gewissermaßen sogar erwartbar. Aber dass die altehrwürdige SPD in diesen Chor einstimmt, kann einen wirklich nur verwundern. Auch bei uns gibt es zahlreiche Ideen dazu, wie man diese Geschäftsordnung verbessern kann. Wir sind sehr für mehr Transparenz, wir haben auch Ideen, die darüber hinausgehen. Aber dass man versucht, in der konstituierenden Sitzung in ein komplexes Regelwerk einzusteigen, ohne vernünftige Beratung, ohne Austausch, ohne Arbeit am Detail leisten zu können, zeigt doch in Wahrheit, dass es Ihnen gar nicht um die Sache geht. Es geht Ihnen in Wahrheit doch um etwas ganz anderes. Wir wissen auch, worum es Ihnen dabei geht. Es geht Ihnen um nichts anderes – Sie haben zum Teil wortgleiche Initiativen der Fraktion Bündnis 90/Die Grünen übernommen –, als die Grünenkollegen in eine peinliche Situation zu bringen. Sie wollen erzwingen, dass die ihre eigenen Anträge ablehnen. Das ist offen gestanden für die konstituierende Sitzung des Deutschen Bundestages eine ganz kleine parteipolitische Münze. Das hat dieses Hohe Haus nicht verdient, meine Damen und Herren. Liebe Kolleginnen und Kollegen, ich kann Ihnen zum heutigen Tage nicht sagen, ob die Fraktion der Freien Demokraten eine Regierung tragen wird oder den Gang in die Opposition antreten wird. Aber ich kann Ihnen eines sagen: Wenn wir den Gang in die Opposition antreten, dann nicht mit solcher Effekthascherei. Das hat das deutsche Volk und das haben die Wähler nicht verdient. Das Wort hat jetzt die Kollegin Britta Haßelmann von Bündnis 90/Die Grünen. 

\textbf{Kommentare:}

Name: Marco Buschmann, Partei: FDP, Fraktion: FDP

\begin{figure}[!ht]
\includegraphics[height=50px]{images/MarcoBuschmann.jpg}
\end{figure}


(Beifall bei der FDP sowie bei Abgeordneten der CDU/CSU – Matthias W. Birkwald [DIE LINKE]: Unverschämtheit!)
Name: Marco Buschmann, Partei: FDP, Fraktion: FDP

\begin{figure}[!ht]
\includegraphics[height=50px]{images/MarcoBuschmann.jpg}
\end{figure}


(Widerspruch bei der SPD)
Name: Marco Buschmann, Partei: FDP, Fraktion: FDP

\begin{figure}[!ht]
\includegraphics[height=50px]{images/MarcoBuschmann.jpg}
\end{figure}


(Beifall beim BÜNDNIS 90/DIE GRÜNEN)
Name: Marco Buschmann, Partei: FDP, Fraktion: FDP

\begin{figure}[!ht]
\includegraphics[height=50px]{images/MarcoBuschmann.jpg}
\end{figure}


(Lachen bei der AfD)
Name: Marco Buschmann, Partei: FDP, Fraktion: FDP

\begin{figure}[!ht]
\includegraphics[height=50px]{images/MarcoBuschmann.jpg}
\end{figure}


(Beifall bei der FDP sowie bei Abgeordneten der CDU/CSU)
Name: Marco Buschmann, Partei: FDP, Fraktion: FDP

\begin{figure}[!ht]
\includegraphics[height=50px]{images/MarcoBuschmann.jpg}
\end{figure}


(Beifall bei der FDP, der CDU/CSU, der SPD, der LINKEN und dem BÜNDNIS 90/DIE GRÜNEN)
Name: Marco Buschmann, Partei: FDP, Fraktion: FDP

\begin{figure}[!ht]
\includegraphics[height=50px]{images/MarcoBuschmann.jpg}
\end{figure}


(Petra Pau [DIE LINKE]: Gucken Sie mal in das Grundgesetz!)
Name: Marco Buschmann, Partei: FDP, Fraktion: FDP

\begin{figure}[!ht]
\includegraphics[height=50px]{images/MarcoBuschmann.jpg}
\end{figure}


(Beifall bei der FDP und der CDU/CSU)


\textbf{NLP-Informationen:}

\textit{DDC-Kategorie der Rede:}

Bräuche, Etikette, Folklore

\textit{Named Entities:}

Personen:

Dr. Baumann, Hermann Görings, Ausschüsse, Liebe Kolleginnen, Britta Haßelmann, 

Orte:

Rolle, Deutschland, Hohe Haus, 

Organisationen:

Präsident!, Deutschen Bundestages, Die Linke, SPD, Die Grünen, Deutschen Bundestages, Freien Demokraten, Die Grünen, 

\textit{Sentiment auf Satzebene:}

0.0, 0.5093, 0.5106, 0.4389, -0.3763, 0.296, 0.0, -0.5106, 0.0, 0.0, -0.0516, 0.3182, 0.0, 0.4019, 0.296, 0.6369, 0.3612, 0.5267, -0.1531, 0.0, 0.4404, 0.0, 0.2598, 0.5106, 0.4767, -0.1779, -0.3182, -0.1027, 0.049, 0.8126, 0.0, -0.1027, -0.2755, 0.3182, 
\subsection{Rede ID19100800}

\textbf{Die Rede:}

% Bestehend aus Name, Partei/Fraktion und Bild
Name: Britta Haßelmann, Partei: BÜNDNIS 90/DIE GRÜNEN, Fraktion: BÜNDNIS 90/DIE GRÜNEN

\begin{figure}[!ht]
\includegraphics[height=50px]{images/BrittaHaßelmann.jpg}
\end{figure}


Sehr geehrter Herr Präsident! Meine Damen und Herren! Kommen wir einmal zum Kern der Debatte zurück. Es geht um das Thema Parlamentsrechte. Das, was wir in den letzten vier Jahren erlebt haben, war kein Glanzstück parlamentarischer Auseinandersetzung, wenn damit inhaltliche Debatten gemeint sind. Deshalb ist es gut, wenn wir uns alle wechselseitig daran erinnern – auch die, die heute vielleicht andere Rollen einnehmen, und all die, die noch nicht wissen, welche Rolle sie einnehmen werden –, was für ermüdende und langatmige Debatten wir hier in Zeiten dieser so Großen Koalition geführt haben. Ich kann mich sehr gut an die vielen Vorstöße aus unserer Fraktion erinnern, das Thema „Rechte des Parlaments stärken“ hier mehrheitsfähig zu machen. Wir sind leider immer wieder damit gescheitert; ich nenne nur: das Thema Parlamentskultur, die Frage der Debattenkultur, die Frage nach Transparenz und Öffentlichkeit von Ausschüssen und auch das Fragerecht oder die Regierungsbefragung. Außer mit dem Bundestagspräsidenten a. D., der auf der Tribüne sitzt, und vielleicht Einzelnen aus den Fraktionen gab es keine Übereinstimmung in dieser Frage. Und deshalb ist die Regierungsbefragung so, wie sie immer war, auch heute noch. Da braucht es dringend eine Änderung. Das sehen auch Bündnis 90/Die Grünen so. Meine Damen und Herren, ich wüsste nicht – deshalb kann ich das Gefeixe aus den Reihen der SPD gar nicht verstehen –, warum wir heute einfach Ihrem Antrag oder aber – es liegen ja zwei Anträge vor – dem Antrag der Linken folgen sollten, der wiederum anders ist. Auch wir haben Vorschläge zur Regierungsbefragung und zur Fragestunde gemacht. Im Antrag der SPD finden sich unsere Punkte nicht in Gänze wieder. Warum haben Sie in Ihrem Antrag zum Beispiel auf die Abschaffung der Konsumtionsregel verzichtet? Das ist doch ein wichtiges Instrument, gerade für die Opposition, meine Damen und Herren. Oder warum wollen Sie die Dringlichen Fragen abschaffen? Das ist ein super Instrument für die Opposition; das würde ich mir noch mal überlegen. Die Frage ist auch: Warum wollen Sie eigentlich keine Befragung zu europäischen Themen? Wir haben doch hier im Parlament den Anspruch, europäische Themen zu beraten, meine Damen und Herren. Also jetzt mal Karten auf den Tisch: Warum sind all diese Sachen in Ihrem Antrag nicht enthalten? Warum die Lacherei darüber, dass Sie uns hier heute vorführen wollen? Das ist doch absurd. Ich möchte mit Ihnen darüber diskutieren, warum Sie auf all diese wesentlichen Sachen verzichten wollen. Ich möchte mit der CDU/CSU darüber diskutieren, dass bei diesem Thema – auch bei der CDU/CSU – mal langsam ein bisschen Bewegung aufkommen muss, dass wir hier eine Veränderung brauchen. Herr Buschmann hat ja schon für die FDP erklärt, dass auch sie sich beim Thema Regierungsbefragung und Fragestunde etwas anderes vorstellt. Deshalb habe ich kein Verständnis dafür, liebe Kolleginnen und Kollegen von SPD und Linken, warum Sie meinem Vorschlag im Vor-Ältestenrat nicht gefolgt sind, es zu überweisen, sich alle Vorschläge aller Fraktionen im Deutschen Bundestag anzuschauen und dann sehr zeitnah darüber zu reden, wie wir es wirklich ändern. Sie wollten uns mit der Geschichte heute einfach vorführen, und da machen wir nicht mit – deshalb der Antrag auf Überweisung. Ich meine das sehr ernst und sehr eindrücklich: Ich wünsche mir, dass wir zur Stärkung des Parlamentes etwas verändern, dass wir beim Thema Transparenz etwas verändern. Dazu gehört die Regierungsbefragung, dazu hat unsere Fraktion auch noch weiter gehende Vorstellungen. Ich fände es toll, wir berieten das sehr zeitnah und in Ruhe, meine Damen und Herren. Jetzt noch zum Antrag auf Einsetzung der Ausschüsse. Jan Korte, es ist ja nicht so, dass Sie den Antrag stellen, Ausschüsse einzusetzen. Nein, Sie beantragen, dass vier Ausschüsse in der Geschäftsordnung zwingend festgelegt werden. Wir sollten mal in Ruhe darüber diskutieren, warum es eigentlich diese vier Ausschüsse sein sollen, deren Einsetzung Sie heute fordern, warum es zum Beispiel nicht der Geschäftsordnungs- und der Immunitätsausschuss sind. Es gibt ab heute hier im Haus allein zwei oder drei Mitglieder – das habe ich der Presse entnommen –, deren Angelegenheiten wir im Immunitätsausschuss zwingend diskutieren müssten, nicht hier im Plenum. Ich fände es gut, wenn wir darüber mal in Ruhe redeten. Meine Damen und Herren, ich finde, wir sollten uns im Deutschen Bundestag sehr zeitnah wieder mit der Frage der Änderungen bei Transparenz und Parlamentsrechten befassen, nachdem alle Fraktionen ihre Vorstellungen eingebracht haben, damit wir es dann hoffentlich sehr schnell zu einem Ergebnis führen. Danke. Wir kommen nun zur Abstimmung. Änderungsantrag der Fraktion der SPD auf Drucksache 19/8 mit dem Titel „Weitergeltung von Geschäftsordnungsrecht“ – Fragestunde und Befragung der Bundesregierung. Die Fraktion der SPD wünscht Abstimmung in der Sache. Die Fraktionen der CDU/CSU, des Bündnisses 90/Die Grünen und der FDP wünschen Überweisung an den Ältestenrat. Wir stimmen nach ständiger Übung zuerst über den Antrag auf Ausschussüberweisung ab. Ich frage deshalb: Wer stimmt für den Antrag der drei genannten Fraktionen auf Überweisung in den Ältestenrat? – Gegenstimmen? – Enthaltungen? – Damit ist der Antrag mit den Stimmen der Fraktionen der CDU/CSU, der FDP und des Bündnisses 90/Die Grünen gegen die Stimmen der SPD, der AfD und der Linken angenommen. Damit stimmen wir heute über den Antrag auf Drucksache 19/8 nicht in der Sache ab. Änderungsantrag der Fraktion der AfD auf Drucksache 19/4 mit dem Titel „Weitergeltung von Geschäftsordnungsrecht (Alterspräsident)“. Die Fraktion der AfD wünscht, über ihren Antrag in der Sache abzustimmen. Die Fraktionen CDU/CSU, Bündnis 90/Die Grünen und FDP wünschen Überweisung an den Ältestenrat. Wir stimmen nach ständiger Übung zuerst über den Antrag auf Überweisung ab. Wer stimmt dem Antrag auf Überweisung zu? – Gegenstimmen? – Enthaltungen? – Der Antrag ist angenommen mit den Stimmen der CDU/CSU-Fraktion, FDP-Fraktion, der Linken und Grünen bei Gegenstimmen bei AfD und SPD und bei Enthaltung der Fraktionslosen. Änderungsantrag der Fraktion der AfD auf Drucksache 19/5 mit dem Titel „Weitergeltung von Geschäftsordnungsrecht (Minderheitenrechte)“. Die Fraktionen der CDU/CSU, des Bündnisses 90/Die Grünen und der FDP wünschen Überweisung an den Ältestenrat. Die Fraktion der AfD wünscht Überweisung an den Ausschuss für Wahlprüfung, Immunität und Geschäftsordnung. Ich lasse zuerst abstimmen über den Antrag der Fraktion der AfD auf Überweisung an den Ausschuss für Wahlprüfung, Immunität und Geschäftsordnung. Wer stimmt für diesen Überweisungsvorschlag? – Wer stimmt dagegen? – Enthaltungen? – Der Überweisungsvorschlag ist mit den Stimmen aller Fraktionen gegen die Stimmen der Fraktion der AfD bei Enthaltung der Fraktionslosen abgelehnt worden. Ich lasse nun abstimmen über den Überweisungsvorschlag der Fraktionen der CDU/CSU, der FDP und des Bündnisses 90/Die Grünen. Wer stimmt für diesen Überweisungsvorschlag? – Gegenstimmen? – Enthaltungen? – Der Überweisungsvorschlag ist angenommen mit den Stimmen der CDU/CSU, der FDP, des Bündnisses 90/Die Grünen und der Fraktion Die Linke bei Gegenstimmen der SPD-Fraktion und Enthaltung der AfD – sowie der beiden fraktionslosen Abgeordneten; die sehe ich immer nicht. Änderungsantrag der Fraktion Die Linke auf Drucksache 19/6 mit dem Titel „Weitergeltung von Geschäftsordnungsrecht (Einsetzung der vom Grundgesetz vorgeschriebenen Ausschüsse in der konstituierenden Sitzung)“. Die Fraktion Die Linke wünscht Abstimmung in der Sache. Die Fraktionen CDU/CSU, FDP und Bündnis 90/Die Grünen wünschen Überweisung an den Ältestenrat. Auch hier stimmen wir nach ständiger Übung zuerst über den Antrag auf Ausschussüberweisung ab. Ich frage deshalb: Wer stimmt für die beantragte Überweisung? – Gegenstimmen? – Enthaltungen? – Die Überweisung ist angenommen mit den Stimmen der Fraktionen CDU/CSU, FDP und Bündnis 90/Die Grünen bei Gegenstimmen von SPD, AfD und der Fraktion Die Linke und – jetzt sehe ich sie – bei Enthaltung der zwei fraktionslosen Abgeordneten. Änderungsantrag der Fraktion Die Linke auf Drucksache 19/7 mit dem Titel „Weitergeltung von Geschäftsordnungsrecht (Fragestunde und Befragung der Bundesregierung)“. Die Fraktion Die Linke wünscht Abstimmung in der Sache. Die Fraktionen CDU/CSU, FDP und Bündnis 90/Die Grünen wünschen Überweisung an den Ältestenrat. Wer stimmt für diesen Überweisungsvorschlag? – Gegenstimmen? – Enthaltungen? – Dann ist der Antrag angenommen mit den Stimmen von CDU/CSU, FDP und Bündnis 90/Die Grünen gegen die Stimmen der anderen Fraktionen, keine Enthaltungen. Antrag der Fraktion der CDU/CSU auf Drucksache 19/1 zur Weitergeltung von Geschäftsordnungsrecht. Wer stimmt für diesen Antrag? – Gegenstimmen? – Enthaltungen? – Dann ist der Antrag mit den Stimmen aller Fraktionen gegen die Stimmen der AfD bei Enthaltung der zwei fraktionslosen Abgeordneten angenommen. 

\textbf{Kommentare:}

Name: Britta Haßelmann, Partei: BÜNDNIS 90/DIE GRÜNEN, Fraktion: BÜNDNIS 90/DIE GRÜNEN

\begin{figure}[!ht]
\includegraphics[height=50px]{images/BrittaHaßelmann.jpg}
\end{figure}


(Beifall beim BÜNDNIS 90/DIE GRÜNEN, bei der CDU/CSU und der FDP)
Name: Britta Haßelmann, Partei: BÜNDNIS 90/DIE GRÜNEN, Fraktion: BÜNDNIS 90/DIE GRÜNEN

\begin{figure}[!ht]
\includegraphics[height=50px]{images/BrittaHaßelmann.jpg}
\end{figure}


(Martin Schulz [SPD]: Jamaika steht!)
Name: Britta Haßelmann, Partei: BÜNDNIS 90/DIE GRÜNEN, Fraktion: BÜNDNIS 90/DIE GRÜNEN

\begin{figure}[!ht]
\includegraphics[height=50px]{images/BrittaHaßelmann.jpg}
\end{figure}


(Beifall beim BÜNDNIS 90/DIE GRÜNEN)
Name: Britta Haßelmann, Partei: BÜNDNIS 90/DIE GRÜNEN, Fraktion: BÜNDNIS 90/DIE GRÜNEN

\begin{figure}[!ht]
\includegraphics[height=50px]{images/BrittaHaßelmann.jpg}
\end{figure}


(Beifall bei der AfD sowie bei Abgeordneten des BÜNDNISSES 90/DIE GRÜNEN)
Name: Britta Haßelmann, Partei: BÜNDNIS 90/DIE GRÜNEN, Fraktion: BÜNDNIS 90/DIE GRÜNEN

\begin{figure}[!ht]
\includegraphics[height=50px]{images/BrittaHaßelmann.jpg}
\end{figure}


(Zuruf des Abg. Dr. Dietmar Bartsch [DIE LINKE])
Name: Britta Haßelmann, Partei: BÜNDNIS 90/DIE GRÜNEN, Fraktion: BÜNDNIS 90/DIE GRÜNEN

\begin{figure}[!ht]
\includegraphics[height=50px]{images/BrittaHaßelmann.jpg}
\end{figure}


(Beifall beim BÜNDNIS 90/DIE GRÜNEN sowie bei Abgeordneten der CDU/CSU, der AfD und der FDP)
Name: Britta Haßelmann, Partei: BÜNDNIS 90/DIE GRÜNEN, Fraktion: BÜNDNIS 90/DIE GRÜNEN

\begin{figure}[!ht]
\includegraphics[height=50px]{images/BrittaHaßelmann.jpg}
\end{figure}


(Petra Pau [DIE LINKE]: Weil es im Grundgesetz steht!)
Name: Britta Haßelmann, Partei: BÜNDNIS 90/DIE GRÜNEN, Fraktion: BÜNDNIS 90/DIE GRÜNEN

\begin{figure}[!ht]
\includegraphics[height=50px]{images/BrittaHaßelmann.jpg}
\end{figure}


(Beifall beim BÜNDNIS 90/DIE GRÜNEN sowie bei Abgeordneten der CDU/CSU und der FDP – Dr. Dietmar Bartsch [DIE LINKE]: Ihr seid noch nicht an der Regierung! Noch nicht!)
Name: Britta Haßelmann, Partei: BÜNDNIS 90/DIE GRÜNEN, Fraktion: BÜNDNIS 90/DIE GRÜNEN

\begin{figure}[!ht]
\includegraphics[height=50px]{images/BrittaHaßelmann.jpg}
\end{figure}


(Jan Korte [DIE LINKE]: Das können wir doch ergänzen!)
Name: Britta Haßelmann, Partei: BÜNDNIS 90/DIE GRÜNEN, Fraktion: BÜNDNIS 90/DIE GRÜNEN

\begin{figure}[!ht]
\includegraphics[height=50px]{images/BrittaHaßelmann.jpg}
\end{figure}


(Claudia Roth [Augsburg] [BÜNDNIS 90/DIE GRÜNEN]: Ja, das habe ich mich auch gefragt!)
Name: Britta Haßelmann, Partei: BÜNDNIS 90/DIE GRÜNEN, Fraktion: BÜNDNIS 90/DIE GRÜNEN

\begin{figure}[!ht]
\includegraphics[height=50px]{images/BrittaHaßelmann.jpg}
\end{figure}


(Michael Grosse-Brömer [CDU/CSU]: Das passiert, wenn man das nicht ordentlich vorbereitet!)
Name: Britta Haßelmann, Partei: BÜNDNIS 90/DIE GRÜNEN, Fraktion: BÜNDNIS 90/DIE GRÜNEN

\begin{figure}[!ht]
\includegraphics[height=50px]{images/BrittaHaßelmann.jpg}
\end{figure}


(Beifall beim BÜNDNIS 90/DIE GRÜNEN sowie bei Abgeordneten der FDP)
Name: Britta Haßelmann, Partei: BÜNDNIS 90/DIE GRÜNEN, Fraktion: BÜNDNIS 90/DIE GRÜNEN

\begin{figure}[!ht]
\includegraphics[height=50px]{images/BrittaHaßelmann.jpg}
\end{figure}


(Jan Korte [DIE LINKE]: Ihr habt noch keinen Koalitionsvertrag!)
Name: Britta Haßelmann, Partei: BÜNDNIS 90/DIE GRÜNEN, Fraktion: BÜNDNIS 90/DIE GRÜNEN

\begin{figure}[!ht]
\includegraphics[height=50px]{images/BrittaHaßelmann.jpg}
\end{figure}


(Beifall beim BÜNDNIS 90/DIE GRÜNEN sowie bei Abgeordneten der SPD und der LINKEN)
Name: Britta Haßelmann, Partei: BÜNDNIS 90/DIE GRÜNEN, Fraktion: BÜNDNIS 90/DIE GRÜNEN

\begin{figure}[!ht]
\includegraphics[height=50px]{images/BrittaHaßelmann.jpg}
\end{figure}


(Beifall beim BÜNDNIS 90/DIE GRÜNEN sowie bei Abgeordneten der SPD und der AfD)
Name: Britta Haßelmann, Partei: BÜNDNIS 90/DIE GRÜNEN, Fraktion: BÜNDNIS 90/DIE GRÜNEN

\begin{figure}[!ht]
\includegraphics[height=50px]{images/BrittaHaßelmann.jpg}
\end{figure}


(Beifall bei Abgeordneten des BÜNDNISSES 90/DIE GRÜNEN – Michael Grosse-Brömer [CDU/CSU]: Gute Frage!)


\textbf{NLP-Informationen:}

\textit{DDC-Kategorie der Rede:}

Politikwissenschaft

\textit{Named Entities:}

Personen:

Meine Damen, Buschmann, Jan Korte, Ausschüsse, 

Orte:

Rolle, Warum, Plenum, 

Organisationen:

Die Grünen, SPD, SPD, Dringlichen Fragen, CDU/CSU, CDU/CSU, FDP, SPD, Linken, Deutschen Bundestag, Deutschen Bundestag, Danke, SPD, SPD, CDU/CSU, Die Grünen, FDP, CDU/CSU, FDP, Die Grünen, SPD, Linken, AfD, AfD, CDU/CSU, Die Grünen, FDP, CDU, FDP-Fraktion, Linken, AfD, SPD, AfD, CDU/CSU, Die Grünen, FDP, AfD, CDU/CSU, FDP, Die Grünen, CDU/CSU, FDP, Die Grünen, Die Linke, SPD-Fraktion, Die Linke, Die Linke, CDU/CSU, FDP, Die Grünen, CDU/CSU, FDP, Die Grünen bei Gegenstimmen, SPD, Die Linke, Die Linke, Die Linke, CDU/CSU, FDP, Die Grünen, CDU/CSU, FDP, Die Grünen, CDU/CSU, Abgeordneten, 

\textit{Sentiment auf Satzebene:}

0.0, 0.0, 0.0, 0.0, -0.34, 0.8608, 0.5256, -0.4019, 0.0, -0.2411, 0.0, 0.0, 0.3182, 0.2335, 0.0, 0.0, -0.4404, 0.3182, -0.1779, 0.4588, 0.1779, 0.0, 0.2023, 0.0, 0.0, 0.0, -0.2023, 0.2732, 0.128, 0.2023, 0.6486, 0.3182, 0.0, 0.0, 0.0, 0.7579, 0.0, 0.0, 0.0, 0.5106, 0.0, 0.7506, 0.6478, 0.4404, 0.25, 0.0, 0.25, 0.3182, 0.0, 0.0, 0.0, 0.0, 0.0, 0.3182, 0.0, 0.0, 0.0, 0.3182, 0.0, 0.0, 0.0, 0.0, 0.0, 0.0, 0.3182, 0.0, 0.128, 0.0, 0.0, 0.0, -0.3182, 0.4215, 0.0, 0.0, 0.0, 0.3182, 0.0, 0.0, 0.0, 0.0, 0.0, 0.25, 0.3182, 0.0, 0.0, 0.0, 0.0, 0.3182, 0.0, 0.0, 0.0, 0.0, 0.0, 0.25, 0.3182, 0.0, 0.0, 0.0, 0.3182, 0.0, 0.0, 0.0, 0.0, 0.0, 0.0, 0.0, 
\section{Tagesordnungspunkt3}
\subsection{Rede ID19100900}

\textbf{Die Rede:}

% Bestehend aus Name, Partei/Fraktion und Bild
Name: Volker Kauder, Partei: CDU, Fraktion: CDU

\begin{figure}[!ht]
\includegraphics[height=50px]{images/VolkerKauder.jpg}
\end{figure}


Die CDU/CSU-Bundestagsfraktion schlägt Herrn Dr. Wolfgang Schäuble vor. 

\textbf{Kommentare:}

Name: Volker Kauder, Partei: CDU, Fraktion: CDU

\begin{figure}[!ht]
\includegraphics[height=50px]{images/VolkerKauder.jpg}
\end{figure}


(Beifall bei der CDU/CSU und der FDP sowie bei Abgeordneten des BÜNDNISSES 90/DIE GRÜNEN)


\textbf{NLP-Informationen:}

\textit{DDC-Kategorie der Rede:}

Soziallehre, Ekklesiologie

\textit{Named Entities:}

Personen:

Herrn Dr. Wolfgang Schäuble, 

Orte:



Organisationen:

CDU, 

\textit{Sentiment auf Satzebene:}

0.0, -0.4767, 
\subsection{Rede ID19101000}

\textbf{Die Rede:}

% Bestehend aus Name, Partei/Fraktion und Bild
Name: Hermann Otto Solms, Partei: FDP, Fraktion: FDP

\begin{figure}[!ht]
\includegraphics[height=50px]{images/Hermann OttoSolms.jpg}
\end{figure}


Der Kollege Dr. Wolfgang Schäuble ist vorgeschlagen. – Ich darf Sie fragen: Sind Sie bereit, zu kandidieren? 

\textbf{Kommentare:}



\textbf{NLP-Informationen:}

\textit{DDC-Kategorie der Rede:}

Literaturen germanischer Sprachen; Deutsche Literatur

\textit{Named Entities:}

Personen:



Orte:



Organisationen:



\textit{Sentiment auf Satzebene:}

0.0, 0.2023, 
\subsection{Rede ID19101100}

\textbf{Die Rede:}

% Bestehend aus Name, Partei/Fraktion und Bild
Name: Wolfgang Schäuble, Partei: CDU, Fraktion: CDU

\begin{figure}[!ht]
\includegraphics[height=50px]{images/WolfgangSchäuble.jpg}
\end{figure}


Ja, Herr Präsident. 

\textbf{Kommentare:}



\textbf{NLP-Informationen:}

\textit{DDC-Kategorie der Rede:}

Bräuche, Etikette, Folklore

\textit{Named Entities:}

Personen:



Orte:



Organisationen:



\textit{Sentiment auf Satzebene:}

0.0, 
\subsection{Rede ID19101200}

\textbf{Die Rede:}

% Bestehend aus Name, Partei/Fraktion und Bild
Name: Hermann Otto Solms, Partei: FDP, Fraktion: FDP

\begin{figure}[!ht]
\includegraphics[height=50px]{images/Hermann OttoSolms.jpg}
\end{figure}


Das ist der Fall. – Ich sehe keine weiteren Vorschläge. Dann bitte ich jetzt um Ihre Aufmerksamkeit für einige Hinweise zum Wahlverfahren – das sind technische Hinweise, die aber wichtig sind, damit Sie an der Wahl erfolgreich teilnehmen können –: Die Wahl findet mit verdeckten Stimmkarten, also geheim, statt. Gewählt ist, wer die Stimmen der Mehrheit der Mitglieder des Bundestages, also mindestens 355 Stimmen, erhält. Für diese Wahl und für die spätere Wahl der Vizepräsidentinnen und Vizepräsidenten benötigen Sie Ihre Wahlausweise aus den Stimmkartenfächern in der Lobby. Für die Wahl des Präsidenten sind der Wahlausweis und die Stimmkarte gelb. Bitte kontrollieren Sie, ob dieser Wahlausweis Ihren Namen trägt. Die gelbe Stimmkarte und den amtlichen Wahlumschlag erhalten Sie nach Aufruf Ihres Namens von den Schriftführerinnen und Schriftführern an den Ausgabetischen hier oben neben den Wahlkabinen. Nachdem Sie die Stimmkarte in einer Wahlkabine gekennzeichnet und dort in den Wahlumschlag gelegt haben, gehen Sie bitte zu den Wahlurnen hier vor dem Rednerpult. Sie dürfen Ihre Stimmkarte nur in der Wahlkabine ankreuzen und müssen ebenfalls noch in der Wahlkabine die Stimmkarte in den Umschlag legen. Die Schriftführerinnen und Schriftführer sind verpflichtet, jeden, der seine Stimmkarte außerhalb der Wahlkabine kennzeichnet oder in den Umschlag legt, zurückzuweisen. Die Stimmabgabe kann in diesem Falle jedoch vorschriftsmäßig wiederholt werden. Gültig sind nur Stimmkarten mit einem Kreuz bei „Ja“, „Nein“ oder „Enthalte mich“. Ungültig sind Stimmen auf nichtamtlichen Stimmkarten sowie Stimmkarten, die mehr als ein Kreuz, kein Kreuz, andere Namen oder Zusätze enthalten. Bevor Sie die Stimmkarte in eine der Wahlurnen werfen, übergeben Sie bitte Ihren Wahlausweis einer der Schriftführerinnen oder einem der Schriftführer an der Wahlurne. Der Nachweis der Teilnahme an der Wahl kann nur durch die Abgabe des Wahlausweises erbracht werden. Ich bitte jetzt die eingeteilten Schriftführerinnen und Schriftführer, die vorgesehenen Plätze einzunehmen. – Die beiden Schriftführer neben mir werden nun Ihre Namen in alphabetischer Reihenfolge aufrufen. Ich bitte Sie, den Namensaufruf zu verfolgen und sich nach dem Aufruf Ihres Namens zur Entgegennahme der Stimmkarte zu den Ausgabetischen vor den Wahlkabinen zu begeben. Haben alle Schriftführerinnen und Schriftführer die Plätze eingenommen? – Ich eröffne die Wahl und bitte, mit dem Aufruf der Namen zu beginnen. Der Namensaufruf ist beendet. Ich darf fragen, ob alle Mitglieder des Hauses, auch die Schriftführerinnen und Schriftführer, ihre Stimme abgegeben haben. – Nein, da kommt noch jemand. Haben jetzt alle Mitglieder des Hauses ihre Stimme abgegeben? – Das scheint der Fall zu sein. Ich schließe die Wahl und bitte die Schriftführerinnen und Schriftführer, mit der Auszählung zu beginnen. Zur Auszählung unterbreche ich die Sitzung für etwa 30 Minuten. Der Wiederbeginn der Sitzung wird rechtzeitig durch Klingelsignal angekündigt. Die unterbrochene Sitzung ist wieder eröffnet. Liebe Kolleginnen und Kollegen, nehmen Sie bitte Platz, damit wir fortfahren können. Ich gebe das Ergebnis der Wahl des Präsidenten des 19. Deutschen Bundestages bekannt: abgegebene Stimmen 705, ungültige Stimmen 1, gültige Stimmen 704. Mit Ja haben gestimmt 501 Kolleginnen und Kollegen. Mit Nein haben gestimmt 173 Kolleginnen und Kollegen, Enthaltungen gab es 30. Herr Dr. Wolfgang Schäuble hat die erforderliche Mehrheit erhalten und ist zum Präsidenten des 19. Deutschen Bundestages gewählt. Ich frage Sie, Herr Schäuble: Nehmen Sie die Wahl an? Herr Präsident, ich nehme die Wahl an. Herr Präsident, ich darf Ihnen im Namen des ganzen Hauses sehr herzlich zu Ihrer Wahl gratulieren. Ich wünsche Ihnen eine glückliche Hand. Verstand muss ich Ihnen nicht wünschen, den haben Sie sowieso. Ich bitte Sie jetzt, das Amt zu übernehmen. Vielen Dank. Tagesordnungspunkt 4: 

\textbf{Kommentare:}

Name: Wolfgang Schäuble, Partei: CDU, Fraktion: CDU

\begin{figure}[!ht]
\includegraphics[height=50px]{images/WolfgangSchäuble.jpg}
\end{figure}


(Beifall bei der CDU/CSU, der SPD, der FDP, der LINKEN und dem BÜNDNIS 90/DIE GRÜNEN sowie bei Abgeordneten der AfD)
Name: Wolfgang Schäuble, Partei: CDU, Fraktion: CDU

\begin{figure}[!ht]
\includegraphics[height=50px]{images/WolfgangSchäuble.jpg}
\end{figure}


(Beifall bei der CDU/CSU, der SPD, der FDP, der LINKEN und dem BÜNDNIS 90/DIE GRÜNEN – Abgeordnete aller Fraktionen gratulieren dem Präsidenten)
Name: Hermann Otto Solms, Partei: FDP, Fraktion: FDP

\begin{figure}[!ht]
\includegraphics[height=50px]{images/Hermann OttoSolms.jpg}
\end{figure}


(Unterbrechung von 12.42 bis 13.09 Uhr)
Name: Hermann Otto Solms, Partei: FDP, Fraktion: FDP

\begin{figure}[!ht]
\includegraphics[height=50px]{images/Hermann OttoSolms.jpg}
\end{figure}


(Beifall bei der CDU/CSU, der SPD, der FDP, der LINKEN und dem BÜNDNIS 90/DIE GRÜNEN)
Name: Hermann Otto Solms, Partei: FDP, Fraktion: FDP

\begin{figure}[!ht]
\includegraphics[height=50px]{images/Hermann OttoSolms.jpg}
\end{figure}


(Anhaltender Beifall bei der CDU/CSU, der SPD, der FDP, der LINKEN und dem BÜNDNIS 90/DIE GRÜNEN)
Name: Hermann Otto Solms, Partei: FDP, Fraktion: FDP

\begin{figure}[!ht]
\includegraphics[height=50px]{images/Hermann OttoSolms.jpg}
\end{figure}


(Namensaufruf und Wahl)


\textbf{NLP-Informationen:}

\textit{DDC-Kategorie der Rede:}

Bräuche, Etikette, Folklore

\textit{Named Entities:}

Personen:

Zusätze, Liebe Kolleginnen, 

Orte:

Lobby, Ausgabetischen, Wahlurne, Ausgabetischen, Namensaufruf, Klingelsignal, Nein, Verstand, 

Organisationen:

Bundestages, Wahlumschlag, 

\textit{Sentiment auf Satzebene:}

0.0, 0.0, 0.836, -0.2263, 0.1779, 0.0, 0.0, 0.0, 0.1779, 0.0, 0.0, 0.0, -0.2263, 0.34, -0.2732, 0.0, -0.3818, 0.0, 0.0, 0.0, 0.25, 0.0, -0.1531, 0.0, 0.0, 0.0, 0.0, -0.2263, -0.3182, 0.0, 0.0, 0.6908, 0.4215, 0.0, 0.1779, 0.0, 0.0, 0.7902, 0.802, 0.3182, 0.0, 0.0, 0.0, 
\section{Tagesordnungspunkt4}
\subsection{Rede ID19101300}

\textbf{Die Rede:}

% Bestehend aus Name, Partei/Fraktion und Bild
Name: Hermann Otto Solms, Partei: FDP, Fraktion: FDP

\begin{figure}[!ht]
\includegraphics[height=50px]{images/Hermann OttoSolms.jpg}
\end{figure}


Herr Bundespräsident! Verehrte Kolleginnen und Kollegen! – Muss ich selber drücken? Aller Anfang ist schwer; also fangen wir noch einmal von vorne an. Herr Bundespräsident! Liebe Kolleginnen und Kollegen! Meine sehr verehrten Damen und Herren! Ich habe zunächst zu danken. Ich danke Ihnen für das Vertrauen, das Sie mir mit der Wahl zum Bundestagspräsidenten entgegenbringen. Ich danke Hermann Otto Solms. Mit seiner langen parlamentarischen Erfahrung hat er die von mir übernommene Aufgabe, diesen 19. Deutschen Bundestag als dienstältester Abgeordneter zu eröffnen, mit großer Umsicht wahrgenommen. Und er hat die Herausforderungen für unser Parlament klar umrissen. Ich möchte den vielen ausgeschiedenen Kolleginnen und Kollegen danken. Sie schauen zum Teil auf jahrzehntelanges parlamentarisches Wirken zurück. Ich nenne stellvertretend Heinz Riesenhuber. Er war gleich zweimal Alterspräsident, bei den konstituierenden Sitzungen der beiden zurückliegenden Legislaturperioden. Ich danke aus dem Präsidium des 18. Deutschen Bundestages den ausgeschiedenen Vizepräsidenten Edelgard Bulmahn und Johannes Singhammer. Ich danke natürlich auch den beiden Vizepräsidentinnen, die dem nächsten Präsidium vermutlich nicht angehören werden. Damit nehme ich jetzt die Wahl vorweg; das ist ein bisschen schwierig. Vor allen Dingen aber, meine sehr verehrten Damen und Herren, möchte ich Norbert Lammert danken. Er war zwölf Jahre ein großartiger Bundestagspräsident. Lieber Herr Lammert, Sie hatten eine ganz besondere Begabung als Redner, und Sie hatten immer klare Vorstellungen davon, was dieses Parlament leisten soll und was es leisten kann, wenn es denn will. Liebe Kolleginnen und Kollegen, ich freue mich auf die neue Aufgabe. Im Parlament schlägt das Herz unserer Demokratie. Ich freue mich auf die Zusammenarbeit mit Ihnen, liebe Kolleginnen und Kollegen, wie mit allen Mitarbeiterinnen und Mitarbeitern, die diesem Haus dienen. Ich bin Parlamentarier aus Leidenschaft. Ich habe meine Abgeordnetentätigkeit immer als hohe Verantwortung und das Mandat als meine demokratische Legitimation verstanden. Ich habe im Übrigen im Deutschen Bundestag beides erlebt: Abgeordneter zu sein in der Opposition wie in einer Regierungsfraktion. Zunächst war ich zehn Jahre in der Opposition. Als ich 1972 zum ersten Mal als Abgeordneter im Deutschen Bundestag saß, wurde um die Ostverträge gestritten – mit leidenschaftlichen Debatten, damals in Bonn. Die Stimmung war aufgeladen. Überhaupt prägte seinerzeit eine extrem spannungsvolle Atmosphäre dieses Land. Die Gesellschaft der Bundesrepublik hatte sich seit Mitte der 60er-Jahre in einem bis dahin nicht gekannten Maße politisiert, mobilisiert und polarisiert. Geschadet hat es nicht, genauso wenig wie die Erregung Anfang der 80er-Jahre. Da war ich Abgeordneter in der großen Regierungsfraktion, als es etwa um den NATO-Doppelbeschluss ging. Sieben Jahre später fiel dann die Mauer. Veränderung war also immer, und vieles wird im Übrigen in der Rückschau anders bewertet als mitten im Streit. Auch deshalb, also weil ich aus eigenem Erleben weiß, dass Erregung und Krisengefühle so neu nicht wirklich sind, sehe ich mit Gelassenheit den Auseinandersetzungen entgegen, die wir in den kommenden Jahren führen werden und die wir im Parlament zu führen haben, stellvertretend für die Gesellschaft, aus der heraus wir gewählt sind. Denn diese Gesellschaft müssen wir nicht nur in ihrem Grundkonsens, sondern auch in ihrer Vielheit und Verschiedenheit repräsentieren. Wir dürfen das eine nicht gegen das andere ausspielen. In einem demokratischen Gemeinwesen ist kein Thema es wert, über den Streit das Gemeinsame in Vergessenheit geraten zu lassen. 289 Abgeordnete ziehen heute erstmals ins Parlament ein – das sind gut 40 Prozent aller Mitglieder dieses Hauses. Selten unterschied sich ein Bundestag so sehr von seinem Vorgänger wie dieser. Sieben Parteien und sechs Fraktionen – so viele gab es seit 60 Jahren nicht mehr. Diese neue Konstellation hier im Haus spiegelt die Veränderungen wider, die unsere Gesellschaft erlebt: Verunsicherungen wachsen angesichts des raschen Wandels durch Globalisierung und Digitalisierung. Zusammenhänge lösen sich auf, Zugehörigkeiten brechen auf und neue entstehen. Alte Gewissheiten und Identitäten werden infrage gestellt, und neue, vermeintliche Gewissheiten werden in Stellung gebracht gegen zunehmende Sorgen und Zweifel. Das menschliche Bedürfnis nach Geborgenheit in vertrauten Lebensräumen trifft auf eine zunehmend als ungemütlich empfundene Welt voller Konflikte, Krisen, Kriege und medial präsentem Schrecken. Vor diesem Hintergrund verschärft sich die Tonlage der gesellschaftlichen Debatten. All das können wir übrigens vielerorts in Europa beobachten. Mit dem ungeheuer schnellen gesellschaftlichen Wandel, den wir erleben, geht eine Fragmentierung unserer Debatten und Aufmerksamkeiten einher. Das stellt die politische Ordnung, die demokratischen Institutionen und Verfahren vor große Herausforderungen. Jedem erscheint etwas anderes wichtig. Jeder scheint gelegentlich nur noch seine eigenen Probleme wahrzunehmen. Es gibt nicht mehr das eine Thema. Das Überhandnehmen von Möglichkeiten und Optionen kann auch überfordern. Über dieses „Unbehagen im Kapitalismus“ hat Uwe Jean Heuser schon 2000 geschrieben. Wie alles ist auch Freiheit durch Übermaß gefährdet. Deswegen müssen wir immer wieder die richtige Balance auch im Umgang mit Freiheit lernen. Hinzu kommt der Wandel der Medien und ihrer Nutzung durch die Veränderungen in der Informationstechnologie. Die Zersplitterung in viele Teilöffentlichkeiten führt dazu, dass uns eine erkennbar gemeinsame Sicht auf politische Prioritäten verloren geht. Da kann dieses Parlament ein Ort der Bündelung, der Fokussierung, der Konzentration auf die wichtigen Fragen unserer gesellschaftlichen Zukunft in Deutschland wie in Europa sein. Wir Abgeordnete, liebe Kolleginnen und Kollegen, sind für die Mitbürger im Wahlkreis manchmal fast eine Art Ombudsmann. Mit unserer Arbeit und unseren Begegnungen vor Ort vermitteln wir diese Wirklichkeit auf die Ebene der Bundespolitik. Unsere Vielzahl an Erfahrungen und Qualifikationen aus beruflicher, sozialer, ehrenamtlicher Tätigkeit bildet eine ganze Menge Expertise. Vielleicht wissen und fühlen wir Abgeordnete durch unsere Verwurzelung bei den Menschen manchmal besser als die Forschungsinstitute, was die Menschen wirklich bewegt. Zugleich sind wir alle, wie Artikel 38 unseres Grundgesetzes sagt, Abgeordnete des ganzen Volkes. Dazu müssen wir diese Vielzahl von Interessen, Meinungen, Befindlichkeiten mit den Begrenztheiten und der Endlichkeit der Realität zusammenbringen, und das zwingt zu Kompromissen und zu Entscheidungen durch Mehrheit. Je besser das gelingt, umso weniger fühlen sich Menschen in der demokratischen Wirklichkeit zurückgelassen. Immanuel Kant, dem wir viele Gedanken von Rechtsstaat und Republik verdanken, hat gesagt – ich drücke es halb mit meinen Worten aus –: Handle stets so, dass das Prinzip Deiner Handlung immer auch das Prinzip der Handlungen aller anderen sein könnte, dass es immer auch allgemeines Gesetz sein könnte. – Also: Handle so, dass menschliches Miteinander nicht zusammenbräche, wenn alle so handelten wie Du selbst. Das, verehrte Kolleginnen und Kollegen, gilt gerade auch für Parlamentsabgeordnete, und das ist eine gute Maxime für unser repräsentatives System. Auch die Vertretung partikularer Interessen darf, wie alles, nicht exzessiv werden. Andere Demokratien in der Welt sind da übrigens schon weit auf die abschüssige Bahn geraten. Was aber sehr wohl sein darf und sein muss, ist, dass der parlamentarische Prozess hier im Hause sichtbar macht, wie schwierig sowohl die Durchsetzung als auch der Ausgleich von Interessen in einer liberalen Demokratie sind. Da darf Streit nicht nur sein; das geht nur über Streit. Den müssen wir führen, und den müssen wir aushalten, ertragen. Demokratischer Streit ist notwendig, aber es ist ein Streit nach Regeln, und es ist mit der Bereitschaft verbunden, die demokratischen Verfahren zu achten und die dann und so zustandegekommenen Mehrheitsentscheidungen nicht als illegitim oder verräterisch oder sonst wie zu denunzieren, sondern die Beschlüsse der Mehrheit zu akzeptieren. Das ist parlamentarische Kultur. Und da kommt es dann auch auf den Stil an, in dem wir uns hier streiten und in dem wir füreinander Respekt signalisieren können. Es gab in den vergangenen Monaten in unserem Land Töne der Verächtlichmachung und Erniedrigung. Ich finde, das hat keinen Platz in einem zivilisierten Miteinander. Die überwältigende Mehrheit der Bürgerinnen und Bürger in diesem Land will ein zivilisiertes Miteinander. In aufgewühlten Zeiten wie unseren wächst das Bedürfnis nach Formen des Verhaltens, über die man lange nicht mehr geredet hat, weil man sie als selbstverständlich ansah. Es wird wieder über Anstand gesprochen – sogar Bücher werden darüber geschrieben und kommen auf die Bestsellerlisten –, und es wird auch über die Frage gesprochen, wie wir in der Gesellschaft miteinander umgehen sollen: Respekt füreinander haben, nicht jeden persönlichen Spielraum maximal ausnutzen, ein offenes Ohr haben für die Argumente des anderen, ihn anerkennen mit seiner anderen Meinung. Es geht um Fairness. Hundertprozentige Gerechtigkeit gibt es nicht, aber Fairness ist möglich in dem Sinne, dass sich möglichst alle angesprochen fühlen und nicht ausgeschlossen bleiben. Die Art, wie wir hier miteinander reden, kann vorbildlich sein für die gesellschaftliche Debatte. Prügeln sollten wir uns hier nicht, wie es ja zum Teil auch in Europa in anderen Parlamenten bisweilen geschieht. Wir sollten das übrigens auch nicht verbal tun. Wir können vielmehr zeigen, dass man sich streiten kann, ohne dass es unanständig wird. Dazu müssen wir zeigen, dass auch ein Bundestag mit sechs Fraktionen schafft, wozu er da ist: Entscheidungen herbeizuführen, die als legitim empfunden werden. Das Parlament besteht aus Abgeordneten, und diese Abgeordneten sind nicht „abgehoben“, wie so gern oberflächlich dahingeredet wird. Wir sind aus der Mitte der Bürgerinnen und Bürger gewählt. Aber niemand vertritt alleine das Volk. So etwas wie Volkswille entsteht überhaupt erst in und mit unseren parlamentarischen Entscheidungen. Deswegen haben wir die Pflicht, diesen Ort wertzuhalten, als Ort des nachvollziehbaren sachlichen wie auch emotionalen Streits – ja, auch Gefühle gehören dazu –, stellvertretend für die Mitbürgerinnen und Mitbürger die Dinge, die alle angehen, argumentativ gegeneinander oder miteinander auszumachen und dann mit Mehrheit zu entscheiden. Wir müssen das Vertrauen in das repräsentative Prinzip wieder stärken. Das ist übrigens keine nur nationale Frage. Die europäischen oder westlichen Werte, die Grundlage unserer verfassungsmäßigen Ordnung sind, wirken vielerorts fragil und erfreuen sich doch zugleich weltweit großer Attraktivität. Freiheit, Rechtsstaatlichkeit, sozialer Zusammenhalt, ökologische Nachhaltigkeit: Ohne Parlamentarismus geht all das nicht. Nach ernsthaftem Streit der Meinungen stellvertretend für alle Bürgerinnen und Bürger Entscheidungen zu treffen: Die befriedende Wirkung, die das hat, wenn es gelingt, brauchen wir überall in der Welt – in einer Welt, wo ja überall immer mehr Menschen nicht nur Anspruch auf wirtschaftliche Teilhabe, sondern auch auf politische Mitsprache erheben. In Zeiten zunehmender Globalisierung heißt das auch, die Kompliziertheit unserer Welt auszuhalten. Aber wir haben zugleich auch die Chance, der Welt, die sich uns nähert, zu zeigen, dass der Parlamentarismus etwas taugt, dass er funktioniert und dass er zu Lösungen für die Probleme und Herausforderungen fähig ist. Norbert Lammert hat immer sehr elegant die Tage, an denen er sprach, danach befragt, was an ihnen in vergangenen Jahren und Jahrhunderten geschah und an was uns das erinnern sollte. Ich will das heute noch einmal im Sinne einer kleinen Hommage tun. Um es chronologisch rückwärts zu machen: Dieser 24. Oktober ist der Tag der Vereinten Nationen. 1945 trat am 24. Oktober die Charta der Vereinten Nationen in Kraft. Am 24. Oktober 1929 endete am Schwarzen Donnerstag die jahrelange Hausse der New Yorker Börse, und es begann die Weltwirtschaftskrise mit all ihren Folgen. Und am 24. Oktober 1648 wurde der Westfälische Frieden zur Beendigung des Dreißigjährigen Krieges unterzeichnet, eines Krieges, an dessen Beginn wir uns kommendes Jahr erinnern. Herfried Münkler hat ihm gerade ein Opus Magnum gewidmet, in dem er zeigt, dass dieser bis heute längste Krieg auf deutschem Boden – zugleich übrigens der erste im vollen Sinne europäische Krieg – uns besser als alle späteren Konflikte die Kriege unserer Gegenwart verstehen lässt. Wer es nicht glaubt, der lese noch einmal im „Simplicissimus“ von ­Grimmelshausen nach. Er ist übrigens in meinem Wahlkreis geschrieben worden. All das, liebe Kolleginnen und Kollegen, erinnert uns an den Charakter der Aufgaben, die vor uns liegen. Es erinnert uns daran, dass wir die Entscheidungen, die wir hier treffen, in weltpolitische Zusammenhänge einzubetten haben. Europa und die Globalisierung: Das ist heute der Rahmen für das, was wir hier debattieren und entscheiden. Das hat nichts mit einem Aufgeben nationaler Selbstbestimmung zu tun, schon gar nichts mit einem Aufgeben des Anspruchs, dass dies hier der Ort ist, an dem immer wieder neu die Souveränität des deutschen Volkes greifbar und wirklich wird. Vielmehr beschreibt es die Aufgabe, der wir gerecht werden müssen, den Weg einer selbstbewussten Einordnung in immer weitere Zusammenhänge zu finden, mit dem Ziel, dazu beizutragen, in dieser Welt unsere Zukunft gestalten zu können. Dass wir uns in solcher Öffnung zur Welt und Einordnung noch selbst erkennen, dass wir bleiben, was wir irgendwie fühlen, das wir sind – im Guten, wie zum Beispiel unserer parlamentarischen Ordnung, wie im Schlechten, das wir als nationale Schicksalsgemeinschaft nicht werden abstreifen können und aus dem wir doch immer wieder neues Gutes zu entwickeln uns bemühen –, dass wir all das bleiben, ohne uns abzuschotten oder uns bequem rauszuhalten, darum, liebe Kolleginnen und Kollegen, geht es. In der Präambel unseres Grundgesetzes von 1949, die wir 1990 im wiedervereinten Deutschland fortgeschrieben haben, heißt es: … von dem Willen beseelt, als gleichberechtigtes Glied in einem vereinten Europa dem Frieden der Welt zu dienen, hat sich das Deutsche Volk kraft seiner verfassungsgebenden Gewalt dieses Grundgesetz gegeben. Dies hier, liebe Kolleginnen und Kollegen, ist der Ort, an dem wir diesem Willen Gestalt geben. Dafür hat uns eine wieder gewachsene Zahl von Bürgerinnen und Bürgern gewählt. Der Trend zur höheren Wahlbeteiligung bei den letzten Landtagswahlen hat sich auch im Bund fortgesetzt. Ich denke, das zeigt, dass Erwartungen gestiegen sind. Wenn wir diese Erwartungen einigermaßen erfüllen, können wir unserem Land einen großen Dienst erweisen. Steigende Erwartungen sind also eine Chance, auch wenn es zur Wahrheit gehört, dass in dieser Welt immer neuer Akteure und immer dichterer Verflechtungen die Realität komplizierter wird und unsere Handlungsspielräume nicht immer nur wachsen. Zwischen beidem müssen wir als Parlament unseren Weg finden. Ich freue mich auf unsere Arbeit hier in den kommenden vier Jahren. Herzlichen Dank. 

\textbf{Kommentare:}

Name: Hermann Otto Solms, Partei: FDP, Fraktion: FDP

\begin{figure}[!ht]
\includegraphics[height=50px]{images/Hermann OttoSolms.jpg}
\end{figure}


(Beifall bei Abgeordneten der AfD sowie des Abg. Michael Grosse-Brömer [CDU/CSU])
Name: Hermann Otto Solms, Partei: FDP, Fraktion: FDP

\begin{figure}[!ht]
\includegraphics[height=50px]{images/Hermann OttoSolms.jpg}
\end{figure}


(Zurufe: Mikro!)
Name: Hermann Otto Solms, Partei: FDP, Fraktion: FDP

\begin{figure}[!ht]
\includegraphics[height=50px]{images/Hermann OttoSolms.jpg}
\end{figure}


(Beifall bei der CDU/CSU, der SPD, der FDP, der LINKEN und dem BÜNDNIS 90/DIE GRÜNEN sowie bei Abgeordneten der AfD)
Name: Hermann Otto Solms, Partei: FDP, Fraktion: FDP

\begin{figure}[!ht]
\includegraphics[height=50px]{images/Hermann OttoSolms.jpg}
\end{figure}


(Beifall bei der CDU/CSU, der SPD, der AfD, der FDP und dem BÜNDNIS 90/DIE GRÜNEN sowie bei Abgeordneten der LINKEN)
Name: Hermann Otto Solms, Partei: FDP, Fraktion: FDP

\begin{figure}[!ht]
\includegraphics[height=50px]{images/Hermann OttoSolms.jpg}
\end{figure}


(Beifall bei der CDU/CSU, der SPD, der FDP, der LINKEN und dem BÜNDNIS 90/DIE GRÜNEN)
Name: Hermann Otto Solms, Partei: FDP, Fraktion: FDP

\begin{figure}[!ht]
\includegraphics[height=50px]{images/Hermann OttoSolms.jpg}
\end{figure}


(Beifall bei der CDU/CSU und der FDP sowie bei Abgeordneten der SPD, der LINKEN und des BÜNDNISSES 90/DIE GRÜNEN)
Name: Hermann Otto Solms, Partei: FDP, Fraktion: FDP

\begin{figure}[!ht]
\includegraphics[height=50px]{images/Hermann OttoSolms.jpg}
\end{figure}


(Heiterkeit und Beifall bei der CDU/CSU, der SPD, der FDP und dem BÜNDNIS 90/DIE GRÜNEN sowie bei Abgeordneten der LINKEN)
Name: Hermann Otto Solms, Partei: FDP, Fraktion: FDP

\begin{figure}[!ht]
\includegraphics[height=50px]{images/Hermann OttoSolms.jpg}
\end{figure}


(Beifall bei der CDU/CSU, der SPD, der LINKEN und dem BÜNDNIS 90/DIE GRÜNEN)
Name: Hermann Otto Solms, Partei: FDP, Fraktion: FDP

\begin{figure}[!ht]
\includegraphics[height=50px]{images/Hermann OttoSolms.jpg}
\end{figure}


(Heiterkeit – Martin Schulz [SPD]: Im „Silbernen Stern“!)
Name: Hermann Otto Solms, Partei: FDP, Fraktion: FDP

\begin{figure}[!ht]
\includegraphics[height=50px]{images/Hermann OttoSolms.jpg}
\end{figure}


(Beifall bei der CDU/CSU, der SPD, der FDP und dem BÜNDNIS 90/DIE GRÜNEN sowie bei Abgeordneten der LINKEN)
Name: Hermann Otto Solms, Partei: FDP, Fraktion: FDP

\begin{figure}[!ht]
\includegraphics[height=50px]{images/Hermann OttoSolms.jpg}
\end{figure}


(Beifall bei der CDU/CSU, der SPD, der FDP und dem BÜNDNIS 90/DIE GRÜNEN sowie bei Abgeordneten der AfD und der LINKEN)
Name: Hermann Otto Solms, Partei: FDP, Fraktion: FDP

\begin{figure}[!ht]
\includegraphics[height=50px]{images/Hermann OttoSolms.jpg}
\end{figure}


(Beifall im ganzen Hause)
Name: Hermann Otto Solms, Partei: FDP, Fraktion: FDP

\begin{figure}[!ht]
\includegraphics[height=50px]{images/Hermann OttoSolms.jpg}
\end{figure}


(Heiterkeit bei Abgeordneten der CDU/CSU, der FDP und des BÜNDNISSES 90/DIE GRÜNEN)


\textbf{NLP-Informationen:}

\textit{DDC-Kategorie der Rede:}

Ethik

\textit{Named Entities:}

Personen:

Hermann Otto Solms, Heinz Riesenhuber, Edelgard Bulmahn, Johannes Singhammer, Norbert Lammert, Lieber Herr Lammert, Liebe Kolleginnen, Uwe Jean Heuser, Immanuel Kant, Du, Verhaltens, Fairness, Norbert Lammert, Herfried Münkler, All, Herzlichen, 

Orte:

Bonn, Rückschau, Europa, Deutschland, Europa, Kompromissen, Europa, Volkswille, lese, Europa, Deutschland, Europa, Deutsche Volk, Landtagswahlen, Bund, 

Organisationen:

Bundespräsident!, Abgeordneter, Deutschen Bundestag, Abgeordneter, Abgeordneter, Deutschen Bundestag, Gesellschaft der Bundesrepublik, Abgeordneter, Bundestag, Alte Gewissheiten, Bundestag, Parlament, Abgeordneten, Abgeordneten, Tag der Vereinten Nationen, New Yorker Börse, 

\textit{Sentiment auf Satzebene:}

0.0, 0.3595, -0.25, 0.0, 0.0, 0.7177, 0.4186, 0.4404, 0.7269, 0.4404, 0.5859, 0.4588, 0.6808, 0.0, 0.0, 0.0, 0.4404, 0.5574, 0.0, -0.2748, 0.796, 0.5719, 0.7845, 0.9001, -0.4767, 0.8402, 0.6124, 0.5719, -0.1027, -0.1027, 0.1027, 0.0, 0.0, 0.2263, 0.0, 0.2732, -0.4767, -0.4588, 0.8334, 0.0, 0.0, 0.0, 0.0, -0.1477, 0.0, 0.4767, 0.0, 0.0, 0.4404, 0.34, 0.5106, -0.3612, -0.8519, 0.0, 0.0, 0.3612, 0.4215, 0.3446, -0.2732, 0.0, 0.3612, 0.0, 0.3818, 0.8481, 0.0, 0.3182, 0.4019, 0.4588, 0.25, 0.8555, 0.7351, 0.0, 0.3818, 0.4767, 0.0, 0.0, 0.821, 0.7351, 0.4019, 0.0, 0.7003, -0.4588, -0.5095, 0.3182, -0.4767, 0.0, 0.4404, -0.5719, 0.6486, 0.802, 0.0, 0.6697, 0.7906, 0.5423, 0.7839, 0.7184, -0.4588, 0.0, -0.1872, 0.5106, -0.046, 0.0, 0.0, 0.0, 0.0, 0.7717, 0.0, 0.8074, 0.8779, -0.3595, 0.0, 0.7717, 0.5563, 0.25, 0.1531, 0.6908, 0.0, -0.6705, -0.9118, 0.0, 0.0, 0.128, 0.0, 0.0, 0.0, 0.841, 0.6908, 0.891, 0.8481, 0.6908, 0.3612, 0.296, 0.25, 0.5118, 0.5368, 0.0, 0.5267, 0.5719, 
\section{Tagesordnungspunkt6}
\subsection{Rede ID19101400}

\textbf{Die Rede:}

% Bestehend aus Name, Partei/Fraktion und Bild
Name: Hans-Peter Friedrich, Partei: CSU, Fraktion: CSU

\begin{figure}[!ht]
\includegraphics[height=50px]{images/Hans-PeterFriedrich.jpg}
\end{figure}


Herr Präsident, ich nehme die Wahl an. Dann beglückwünsche ich Sie, Herr Kollege Friedrich. Herr Kollege Thomas Oppermann, nehmen Sie Wahl an? 

\textbf{Kommentare:}

Name: Hans-Peter Friedrich, Partei: CSU, Fraktion: CSU

\begin{figure}[!ht]
\includegraphics[height=50px]{images/Hans-PeterFriedrich.jpg}
\end{figure}


(Beifall bei der CDU/CSU und der FDP sowie bei Abgeordneten der SPD)


\textbf{NLP-Informationen:}

\textit{DDC-Kategorie der Rede:}

Bräuche, Etikette, Folklore

\textit{Named Entities:}

Personen:

Friedrich, Thomas Oppermann, 

Orte:



Organisationen:



\textit{Sentiment auf Satzebene:}

0.0, 0.0, 0.0, 
\subsection{Rede ID19101500}

\textbf{Die Rede:}

% Bestehend aus Name, Partei/Fraktion und Bild
Name: Thomas Oppermann, Partei: SPD, Fraktion: SPD

\begin{figure}[!ht]
\includegraphics[height=50px]{images/ThomasOppermann.jpg}
\end{figure}


Jawohl, Herr Präsident, ich nehme die Wahl an. Ich beglückwünsche Sie und freue mich auf gute Zusammenarbeit. Herr Kollege Kubicki, nehmen Sie die Wahl an? 

\textbf{Kommentare:}

Name: Thomas Oppermann, Partei: SPD, Fraktion: SPD

\begin{figure}[!ht]
\includegraphics[height=50px]{images/ThomasOppermann.jpg}
\end{figure}


(Beifall bei der SPD)


\textbf{NLP-Informationen:}

\textit{DDC-Kategorie der Rede:}

Bräuche, Etikette, Folklore

\textit{Named Entities:}

Personen:

Kubicki, 

Orte:



Organisationen:



\textit{Sentiment auf Satzebene:}

0.0, 0.8442, 0.0, 
\subsection{Rede ID19101600}

\textbf{Die Rede:}

% Bestehend aus Name, Partei/Fraktion und Bild
Name: Wolfgang Kubicki, Partei: FDP, Fraktion: FDP

\begin{figure}[!ht]
\includegraphics[height=50px]{images/WolfgangKubicki.jpg}
\end{figure}


Herr Präsident, ich nehme die Wahl an. Dann beglückwünsche ich auch Sie im Namen des Hauses und freue mich auf gute Zusammenarbeit. Frau Kollegin Pau, nehmen Sie die Wahl an? 

\textbf{Kommentare:}

Name: Wolfgang Kubicki, Partei: FDP, Fraktion: FDP

\begin{figure}[!ht]
\includegraphics[height=50px]{images/WolfgangKubicki.jpg}
\end{figure}


(Beifall bei der FDP sowie bei Abgeordneten der CDU/CSU)


\textbf{NLP-Informationen:}

\textit{DDC-Kategorie der Rede:}

Bräuche, Etikette, Folklore

\textit{Named Entities:}

Personen:

Kollegin Pau, 

Orte:



Organisationen:



\textit{Sentiment auf Satzebene:}

0.0, 0.8442, 0.0, 
\subsection{Rede ID19101700}

\textbf{Die Rede:}

% Bestehend aus Name, Partei/Fraktion und Bild
Name: Petra Pau, Partei: DIE LINKE., Fraktion: DIE LINKE.

\begin{figure}[!ht]
\includegraphics[height=50px]{images/PetraPau.jpg}
\end{figure}


Herr Präsident, ich nehme die Wahl an und freue mich auf die Zusammenarbeit. Herzlichen Glückwunsch, Frau Pau. Auch ich freue mich auf die Zusammenarbeit. Jetzt frage ich die Frau Kollegin Roth, ob sie die Wahl annimmt. 

\textbf{Kommentare:}

Name: Petra Pau, Partei: DIE LINKE., Fraktion: DIE LINKE.

\begin{figure}[!ht]
\includegraphics[height=50px]{images/PetraPau.jpg}
\end{figure}


(Beifall bei der LINKEN sowie bei Abgeordneten der CDU/CSU, der SPD, der FDP und des BÜNDNISSES 90/DIE GRÜNEN)


\textbf{NLP-Informationen:}

\textit{DDC-Kategorie der Rede:}

Andere Religionen

\textit{Named Entities:}

Personen:

Herzlichen Glückwunsch, Pau, Kollegin Roth, 

Orte:



Organisationen:



\textit{Sentiment auf Satzebene:}

0.7184, 0.8074, 0.7184, 0.0, 
\subsection{Rede ID19101800}

\textbf{Die Rede:}

% Bestehend aus Name, Partei/Fraktion und Bild
Name: Claudia Roth, Partei: BÜNDNIS 90/DIE GRÜNEN, Fraktion: BÜNDNIS 90/DIE GRÜNEN

\begin{figure}[!ht]
\includegraphics[height=50px]{images/ClaudiaRoth.jpg}
\end{figure}


Ja, ich nehme die Wahl mit großer Freude an und freue mich auch auf die Zusammenarbeit. Die Freude ist unübersehbar. Ich beglückwünsche Sie im Namen des ganzen Hauses, Frau Kollegin Roth. Ich wünsche Ihnen allen im Namen des Hauses Glück und Erfolg für das verantwortungsvolle Amt. Da der Kollege Glaser nicht die erforderliche Mehrheit der Mitglieder des Hauses erhalten hat, ist nach der Geschäftsordnung ein zweiter Wahlgang erforderlich. Wir führen diesen zweiten Wahlgang jetzt unmittelbar durch. Für diesen zweiten Wahlgang schlägt die Fraktion der AfD den Kollegen Glaser vor. Auch im zweiten Wahlgang ist nach unserer Geschäftsordnung die Mehrheit der Mitglieder des Bundestages, also 355 Stimmen, erforderlich. Die Wahl ist wiederum geheim. Wahlausweise und Stimmkarte für diesen Wahlgang sind blau. Die Stimmkarte erhalten Sie wieder vor den Wahlkabinen. – Jetzt bitte ich Sie – genauso wie zuvor –, sich die Wahlunterlagen zu besorgen und in den Kabinen Ihr Kreuz zu machen. Wir beginnen mit der Wahl. Sind alle Urnen besetzt? – Das ist der Fall. Dann ist die Wahl eröffnet. Liebe Kolleginnen und Kollegen, darf ich fragen, ob jemand nicht die Möglichkeit hatte, seine Stimme abzugeben? – Das ist offensichtlich nicht der Fall. Dann schließe ich die Wahl. Ich bitte die Schriftführerinnen und Schriftführer, mit der Auszählung zu beginnen. Ich unterbreche die Sitzung für etwa eine halbe Stunde. Der Wiederbeginn wird rechtzeitig durch Klingelsi­gnal angekündigt. (Unterbrechung von 15.55 bis 16.15 Uhr) Liebe Kolleginnen und Kollegen, die unterbrochene Sitzung ist wieder eröffnet. Ich darf Sie bitten, Platz zu nehmen. Ich teile Ihnen das Ergebnis des zweiten Wahlgangs zur Wahl eines Stellvertreters des Präsidenten mit: abgegebene Stimmen 697, ungültige Stimmen 1, gültige Stimmen 696. Mit Ja haben gestimmt 123, mit Nein haben gestimmt 549, Enthaltungen 24. Der Abgeordnete Glaser hat damit die erforderliche Mehrheit nicht erhalten.1  Namensverzeichnis der Teilnehmer an der Wahl siehe Anlage 4 Jetzt schaue ich den Kollegen Baumann an. – Die Fraktion der AfD beantragt, einen dritten Wahlgang durchzuführen. Es gibt keinen neuen Kandidaten. Ich würde Ihnen gerne noch erläutern, wie der dritte Wahlgang abläuft. Nach unserer Geschäftsordnung ist im dritten Wahlgang der Kandidat Herr Glaser gewählt, wenn er die Mehrheit der abgegebenen Stimmen auf sich vereinigt, also die Zahl der Jastimmen größer ist als die Zahl der Neinstimmen. Enthaltungen bleiben insofern unberücksichtigt. Die Wahl ist wiederum geheim. Wahlausweis und Stimmkarte für diesen Wahlgang sind rot. Alles andere, Wahlkabine etc., läuft so ab wie bei den ersten beiden Wahlgängen auch. Ich bitte, die Wahlunterlagen in Empfang zu nehmen und die Urnen zu besetzen. Dazu brauchen wir Schriftführer. – Die Urnen sind besetzt. Ich eröffne den dritten Wahlgang. Liebe Kolleginnen und Kollegen, darf ich fragen, ob noch ein Abgeordneter seine Stimmkarte abgeben möchte, der sie nicht abgegeben hat? – Das ist nicht der Fall. Dann schließe ich den Wahlgang und bitte die Schriftführerinnen und Schriftführer, mit der Auszählung zu beginnen. Ich unterbreche die Sitzung wieder für etwa eine halbe Stunde. Auch diesmal wird durch Klingelzeichen rechtzeitig angekündigt, wenn die Sitzung wieder beginnt. 

\textbf{Kommentare:}

Name: Claudia Roth, Partei: BÜNDNIS 90/DIE GRÜNEN, Fraktion: BÜNDNIS 90/DIE GRÜNEN

\begin{figure}[!ht]
\includegraphics[height=50px]{images/ClaudiaRoth.jpg}
\end{figure}


(Beifall bei Abgeordneten des BÜNDNISSES 90/DIE GRÜNEN)
Name: Claudia Roth, Partei: BÜNDNIS 90/DIE GRÜNEN, Fraktion: BÜNDNIS 90/DIE GRÜNEN

\begin{figure}[!ht]
\includegraphics[height=50px]{images/ClaudiaRoth.jpg}
\end{figure}


(Unterbrechung von 16.39 bis 16.59 Uhr)
Name: Claudia Roth, Partei: BÜNDNIS 90/DIE GRÜNEN, Fraktion: BÜNDNIS 90/DIE GRÜNEN

\begin{figure}[!ht]
\includegraphics[height=50px]{images/ClaudiaRoth.jpg}
\end{figure}


(Beifall beim BÜNDNIS 90/DIE GRÜNEN sowie bei Abgeordneten der CDU/CSU)


\textbf{NLP-Informationen:}

\textit{DDC-Kategorie der Rede:}

Politikwissenschaft

\textit{Named Entities:}

Personen:

Kollegin Roth, Liebe Kolleginnen, Baumann, Liebe Kolleginnen, Klingelzeichen, 

Orte:

Wiederbeginn, Jastimmen, Wahlausweis, 

Organisationen:

AfD, Bundestages, AfD, Abgeordneter, 

\textit{Sentiment auf Satzebene:}

0.8957, 0.5719, 0.0, 0.891, 0.1779, 0.3182, -0.4767, 0.0, 0.0, 0.0, 0.1779, 0.0, 0.0, 0.0, 0.0, 0.0, 0.5559, 0.0772, -0.2263, 0.0, -0.3182, 0.0, 0.0, 0.6908, 0.0, 0.0772, 0.0, 0.0, 0.0, 0.0, 0.0, 0.4404, -0.1779, 0.0, 0.0, 0.0, 0.0, 0.0, 0.0, 0.0, 0.0, 0.0, 0.8126, 0.0, -0.2263, -0.3182, 0.0, 
\subsection{Rede ID19101900}

\textbf{Die Rede:}

% Bestehend aus Name, Partei/Fraktion und Bild
Name: Hermann Otto Solms, Partei: FDP, Fraktion: FDP

\begin{figure}[!ht]
\includegraphics[height=50px]{images/Hermann OttoSolms.jpg}
\end{figure}


Die unterbrochene Sitzung ist wieder eröffnet. Liebe Kolleginnen und Kollegen, ich darf Sie bitten, wieder Platz zu nehmen. Mir liegt das Ergebnis des dritten Wahlgangs vor. Sobald Sie Platz genommen haben, werde ich es Ihnen mitteilen. Im dritten Wahlgang für die Wahl einer Stellvertreterin/eines Stellvertreters des Präsidenten wurden 685 Stimmen abgegeben. 685 Stimmen waren gültig. Mit Ja haben gestimmt 114, mit Nein haben gestimmt 545, Enthaltungen 26. Der Abgeordnete Glaser hat damit nicht die erforderliche Mehrheit erhalten.2 Namensverzeichnis der Teilnehmer an der Wahl siehe Anlage 5 Gemäß § 2 Absatz 3 unserer Geschäftsordnung findet kein weiterer Wahlgang mit einem im dritten Wahlgang erfolglosen Kandidaten statt, es sei denn, ein weiterer Wahlgang wird im Ältestenrat vereinbart. Wird ein neuer Bewerber vorgeschlagen, so ist in ein neues Wahlverfahren einzutreten. Es wäre also wieder ein erster Wahlgang mit den entsprechenden Mehrheitserfordernissen durchzuführen. In beiden Fällen muss aber erst vereinbart werden, an welchem Tag die Wahl durchgeführt werden soll. Die Geschäftsführerinnen und Geschäftsführer der Fraktionen haben mir signalisiert, dass es kein Einvernehmen über die Durchführung weiterer Wahlgänge am heutigen Tag gibt. Damit ist dieser Tagesordnungspunkt abgeschlossen. Damit darf ich Sie bitten, sich zu unserer Nationalhymne von Ihren Sitzen zu erheben. Liebe Kolleginnen und Kollegen, wir sind damit am Schluss unserer heutigen Tagesordnung. Die nächste Sitzung findet in der 47. Kalenderwoche, also in der Woche vom 20. November, statt. Der genaue Tag der Sitzung wird noch bestimmt und Ihnen bekannt gegeben. Bevor ich die Sitzung schließe, darf ich Sie herzlich zu einem kleinen Empfang auf der Fraktionsebene einladen. Die Sitzung ist geschlossen. (Schluss: 17.03 Uhr) 

\textbf{Kommentare:}

Name: Hermann Otto Solms, Partei: FDP, Fraktion: FDP

\begin{figure}[!ht]
\includegraphics[height=50px]{images/Hermann OttoSolms.jpg}
\end{figure}


(Nationalhymne – Beifall)


\textbf{NLP-Informationen:}

\textit{DDC-Kategorie der Rede:}

Soziallehre, Ekklesiologie

\textit{Named Entities:}

Personen:

Liebe Kolleginnen, Wahlgangs, Liebe Kolleginnen, 

Orte:

Mir, 

Organisationen:



\textit{Sentiment auf Satzebene:}

0.0, 0.6908, 0.0, 0.0, 0.0, 0.34, 0.0, 0.0, 0.0, 0.0, 0.7003, 0.0, 0.0, 0.0, 0.0258, 0.0, 0.6908, 0.1779, 0.5719, 0.34, -0.2263, 0.0, 



\end{document}
